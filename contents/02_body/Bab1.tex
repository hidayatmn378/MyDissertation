%\vspace{1.5pc}
\vspace{1.5pc}
\section[Latar Belakang]{Latar Belakang}
\begin{spacing}{1.5}
	Samudera Hindia adalah samudera terbesar ketiga di dunia, meliputi sekitar 19.8\% dari total volume lautan \cite{Eakins2010} dan merupakan lautan yang sangat berpengaruh bagi ekosistem di Bumi. Cakupan wilayah dari Samudera Hindia termasuk di dalamnya Teluk Benggala (\textit{Bay of Bengal} (BoB)), Laut Andaman, Selat Malaka, dan Perairan Aceh. Dengan cakupan wilayah yang begitu luas, Samudera Hindia merupakan penyumbang besar bagi sistem iklim dunia dan oleh karena itu sangat penting untuk dapat diprediksi. Pengembangan model kelautan berusaha untuk menggambarkan iklim global dengan cukup baik disertai dengan pengamatan. Namun, variabilitas spasial dan temporal perlu dipahami untuk prediksi yang lebih baik. Pemanasan matahari dan kekuatan angin bervariasi dalam ruang dan waktu yang akan tercermin dalam variabilitas lapisan campuran laut dan suhu permukaan. Oleh karena itu, fokus utama dari tesis ini adalah peran gaya atmosfer lokal pada variabilitas lapisan campuran dan akibatnya pada suhu permukaan laut.
	
	Beberapa studi observasional dan pemodelan telah dilakukan untuk mempelajari pengaruh interaksi atmosfer-laut terhadap variabilitas suhu permukaan laut (SST), salinitas permukaan laut (SSS), klorofil-a (chl-a), kedalaman lapisan campuran (MLD) dan sirkulasi pada wilayah perairan Samudera Hindia, diantaranya adalah, \cite{Kantha2019} yang meneliti tentang pencampuran turbulen di lapisan atas BoB utara dipengaruhi oleh lapisan dangkal yang menutupi perairan asin teluk, yang dihasilkan dari arus besar air tawar dari sungai-sungai besar yang mengalir dari anak benua Asia dan dari curah hujan di atas teluk selama musim panas. Karena BoB juga berbatasan dengan laut Arab, perbedaan sering terjadi pada musim dingin, yaitu upwelling dan konveksi musim dingin, yang meningkatkan biomassa fitoplankton di Laut Arab, tetapi sangat lemah atau bahkan tidak ada di BoB. 
	
	Demikian pula, masukan nutrisi melalui aliran sungai ke BoB tidak cukup untuk meningkatkan stok fitoplankton di luar perairan \cite{Jyothibabu2021}. BoB memiki keunikan akibat instrusi air tawar dari curah hujan yang tinggi selama musim panas sebagai hasil penetrasi insolasi matahari di kolom air \cite{Kantha2019}. \cite{Srivastava2018} mensimulasikan model tanpa gaya angin dekat permukaan, hasilnya adalah SST (\textit{Sea Surface Temperature}) wilayah tersebut sangat meningkat di semua musim, sedangkan, tanpa adanya gaya radiasi gelombang pendek yang masuk, mereka mendapatkan hasil yang benar-benar berlawanan. Ditemukan bahwa pengaruh pemaksaan fluks air tawar pada SST wilayah tersebut sangat kecil. Ditemukan juga bahwa SSS (\textit{Sea Surface Salinity}) laut Arab dan BoB menurun tanpa adanya gaya angin dekat permukaan dan radiasi gelombang pendek yang masuk, sedangkan di BoB utara meningkat tanpa adanya gaya fluks air tawar \cite{Srivastava2018}.
	
	\section[Rumusan Masalah]{Rumusan Masalah}
	Pada latar belakang, telah diuraikan penelitian-penelitian terkait MLD dan mengapa MLD penting untuk menggambarkan iklim global. Telah dijelaskan pula secara ringkas mengenai hal-hal apa saja yang akan dilakukan dalam penelitian ini. Fokus dari penelitian tesis ini adalah menjawab masalah utama, yaitu
	
	Bagaimana pengaruh parameter meteorologi terhadap kedalaman lapisan campuran (\textit{Mixed Layer Depth}) di Perairan Aceh?
	
	Subpertanyaan berikut akan berkontribusi pada perumusan jawaban atas masalah utama.
	\begin{itemize}
		\item Bagaimana analisis kedalaman lapisan campuran (MLD) di wilayah perairan Aceh dalam 12 bulan pada tahun 2021? 
		\item Bagaimana analisis model iklim untuk parameter-parameter meteorologi \textit{2m air temperature, 2m specific humidity, convective precipitation rate, sea level pressure, wind stress U}, dan \textit{wind stress V} selama 22 tahun, tahun 2000 - 2021?
		\item Bagaimana hubungan parameter meteorologi terhadap analisis kedalaman lapisan campuran (MLD) di wilayah perairan Aceh?
	\end{itemize}
	\section[Tujuan Penelitian]{Tujuan Penelitian}
	
	Tujuan dari penelitian tesis ini adalah mencari tahu pengaruh parameter meteorologi terhadap kedalaman lapisan campuran (\textit{Mixed Layer Depth}) di Perairan Aceh dengan cara menjawab beberapa masalah terkait,
	
	\begin{itemize}
		\item Analisis kedalaman lapisan campuran (MLD) di wilayah perairan Aceh dalam 12 bulan pada tahun 2021.
		\item Analisis model iklim untuk parameter-parameter meteorologi \textit{2m air temperature, 2m specific humidity, convective precipitation rate, sea level pressure, wind stress U}, dan \textit{wind stress V} selama 22 tahun, tahun 2000 - 2021.
		\item Hubungan parameter meteorologi terhadap analisis kedalaman lapisan campuran (MLD) di wilayah perairan Aceh.
	\end{itemize}
	
	\section[Urgensi dan Kebaruan Penelitian]{Urgensi dan Kebaruan Penelitian}

	Sejauh pengamatan kami, studi secara detail terkait 6 parameter meteorologi dan dampaknya terhadap lapisan vertikal di wilayah perairan Aceh belum pernah dilakukan sebelumnya. Oleh karena itu, dirasa penting untuk melakukan penelitian ini guna mengetahui pengaruh paramater meteorologi terhadap kedalaman lapisan campuran (MLD).

	\section[Manfaat Penelitian]{Manfaat Penelitian}
	
	Penelitian ini diharapkan mampu memberikan kontribusi ilmiah dan memperkaya pengetahuan tentang kedalaman lapisan campuran atau MLD. Hal ini karena MLD berperan penting secara iklim fisik dalam hal menentukan interval kisaran temperatur di wilayah laut dan pesisir. Sebagai tambahan, panas yang tersimpan dalam lapisan campuran menyediakan sumber panas yang mendorong variabilitas global seperti El Ni$\tilde{n}$o. MLD juga berperan dalam menentukan tingkatan rata-rata cahaya yang dapat dilihat oleh organisme laut seperti fitoplankton. Selain itu, dari periodesitas model iklim yang diperoleh akan bermanfaat untuk tujuan fishing ground, mitigasi perubahan iklim dan bencana hidro-oseanografi, tata ruang dan konservasi
	laut, dan sumber energi terbarukan. 

	\section[Sistematika Penulisan]{Sistematika Penulisan}

	Tesis ini tersusun atas 5 bab. Bab pertama menjelaskan pendahuluan tentang latar belakang mengapa penelitian ini dilakukan, background masalah yang mendasari, tujuan penelitian, manfaat penelitian, serta kebaruan dari penelitian. Bab kedua berisikan tinjauan pustaka menyangkut ulasan singkat materi penelitian. Bab ketiga membahas tentang metode penelitian yang dilakukan, data yang yang digunakan, serta diagram alir (\textit{flowchart}) dari penelitian. Bab keempat membahas hasil dan pembahasan penelitian. Terakhir, bab kelima membahas tentang kesimpulan dari penelitian.
	
\end{spacing}