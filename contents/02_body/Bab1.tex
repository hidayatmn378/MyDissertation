%\vspace{1.5pc}
\vspace{1.5pc}
\section[Latar Belakang]{Latar Belakang}
\begin{spacing}{1.5}
	Samudera Hindia merupakan salah satu dari tiga samudera terbesar di dunia, dengan volume lautan mencapai sekitar 19.8\% dari total volume lautan di seluruh dunia \cite{Eakins2010}. Karena wilayahnya yang sangat luas, Samudera Hindia memiliki peran penting dalam sistem iklim global dan oleh karena itu sangat penting untuk dapat memprediksi perubahan yang terjadi di dalamnya.
	
	IOD atau \textit{Indian Ocean Dipole} merupakan interaksi anomali antara laut dan atmosfer, yang melibatkan osilasi yang tidak teratur dari temperature permukaan laut (SST) di Samudera Hindia tropis. IOD adalah perbedaan SST antara Samudera Hindia bagian barat tropis ($50^\circ$ E-$70^\circ$ E, $10^\circ$ S-$10^\circ$ N) dan Samudera Hindia bagian timur tropis ($90^\circ$ E-$110^\circ$ E, $10^\circ$ S-$0^\circ$ N) \cite{Shunmugapandi2022,Thushara2020,Sattar2019}. Kekuatan IOD diukur berdasarkan \textit{dipole mode index} (DMI) \cite{Saji1999}. Kekuatan dan frekuensi peristiwa IOD telah terbukti memiliki dampak signifikan pada iklim musim hujan \cite{Qiu2014}, pertanian \cite{Zhang2016}, dan perikanan \cite{Lan2013} di wilayah tersebut. Dengan peningkatan emisi gas rumah kaca antropogenik (dihasilkan oleh aktivitas manusia), kekhawatiran tentang dampak potensial perubahan iklim pada IOD dan pola iklim yang terkait semakin meningkat.
	
	Untuk lebih memahami hubungan antara IOD dan perubahan iklim, sangat penting untuk memodelkan parameter oseanografi dan meteorologi yang menggerakkan IOD. Penelitian ini akan menggunakan aplikasi pemodelan laut untuk mempelajari keterkaitan antara IOD, parameter oseanografi seperti arus laut, temperatur laut, salinitas, MLD, Chl-a, fluks air tawar, dan fluks panas bersih, dan parameter meteorologi seperti tekanan angin dan laju presipitasi. Dengan menganalisis interaksi antara parameter-parameter ini, penelitian ini bertujuan untuk memberikan wawasan tentang mekanisme yang mendorong IOD dan potensi responsnya terhadap perubahan iklim.
	
	Penggunaan aplikasi pemodelan laut semakin penting dalam studi perubahan iklim, karena memungkinkan para peneliti untuk mensimulasikan dan memprediksi perilaku parameter oseanografi dan meteorologi dari waktu ke waktu. Studi tentang pemodelan laut berkaitan dengan pembentukan model dan sifat-sifat sistem di dalamnya. Dalam praktiknya, pembentukan model ini memanfaatkan model numerik dan program komputasi dengan tujuan untuk mengatasi keterbatasan data observasi (\textit{in situ}), juga untuk alasan efektifitas serta efisiensi biaya dan waktu yang digunakan. Penelitian sebelumnya mengembangkan model numerik dengan menggunakan persamaan Navier-Stokes untuk memodelkan sirkulasi arus laut dan menganalisis variabel hidrodinamika laut lainnya. Sebagai contoh, simulasi arus laut di perairan Indonesia akibat gaya pembangkit angin menggunakan model persamaan Navier-Stokes 2 dimensi \cite{Rizal2018,Ikhwan2019,Haditiar2019}. Dalam penelitian lain, model persamaan Navier-Stokes 3 dimensi digunakan untuk mengkaji sirkulasi arus pasang surut baroklinik M2 dan hidrodinamika laut yang berasal dari fenomena El Nino \cite{Rizal2010,Haditiar2020,Ikhwan2021}.
	
	Model-model ini dapat digunakan untuk mempelajari dampak perubahan iklim pada laut dan ekosistem yang terkait, serta untuk mengembangkan strategi mitigasi dan adaptasi terhadap perubahan tersebut. Dalam penelitian ini, model laut akan digunakan secara jangka panjang untuk mempelajari IOD, yang diharapkan dapat berkontribusi pada pemahaman kita tentang interaksi kompleks antara laut dan atmosfer di wilayah Samudera Hindia. 
%	Selain itu, penelitian ini akan mengeksplorasi dampak potensial dari perubahan iklim pada IOD dan pola iklim yang terkait. Dengan menganalisis data iklim historis dan proyeksi, studi ini akan mengevaluasi potensi perubahan frekuensi dan intensitas kejadian IOD dalam berbagai skenario perubahan iklim. Penelitian ini juga akan mengevaluasi dampak potensial dari perubahan ini pada iklim musim hujan dan sektor perikanan di wilayah tersebut.
	
	\section[Rumusan Masalah]{Rumusan Masalah}
	Secara keseluruhan, masalah utama dari penelitian ini adalah menyelidiki hubungan antara IOD, parameter oseanografi, dan meteorologi seperti arus laut, temperatur laut, salinitas, MLD, Chl-a, fluks air tawar, fluks panas bersih, laju presipitasi, dan tekanan angin dengan menggunakan aplikasi pemodelan laut, serta mengevaluasi dampak potensial perubahan pada IOD dan pola iklim yang terkait. Dengan memahami mekanisme yang mendorong IOD dan dampaknya terhadap perubahan iklim, penelitian ini diharapkan akan memberikan kontribusi pada ilmu pengetahuan, yaitu pemahaman tentang interaksi kompleks antara laut dan atmosfer serta memberikan strategi untuk mitigasi dan adaptasi terhadap dampak perubahan iklim di wilayah Samudera Hindia.

	\section[Tujuan Penelitian]{Tujuan Penelitian}
	
	Tujuan umum dari penelitian ini adalah menyelidiki hubungan antara IOD dengan parameter oseanografi dan meteorologi seperti arus laut, temperatur laut, salinitas, MLD, Chl-a, fluks air tawar, fluks panas bersih, laju presipitasi, dan tekanan angin, serta dampaknya terhadap ekosistem laut dengan cara menjawab beberapa tujuan khusus berikut.
	
	\begin{itemize}
		\item Menggambarkan peta IOD, arus laut, temperatur laut, salinitas, MLD, Chl-a, fluks air tawar, fluks panas bersih, laju presipitasi, dan tekanan angin di Samudera Hindia.
		\item Menggambarkan dan menganalisis model musiman untuk parameter-parameter oseanografi dan meteorologi, seperti arus laut, temperatur laut, salinitas, MLD, Chl-a, fluks air tawar, fluks panas bersih, laju presipitasi, dan tekanan angin di Samudera Hindia.
		\item Melakukan analisis korelasi untuk mengukur kekuatan dan arah dari hubungan antara IOD dengan parameter-parameter oseanografi dan meteorologi.
	\end{itemize}
	\section[Urgensi Penelitian]{Urgensi Penelitian}

	Samudera Hindia merupakan rumah bagi populasi biota laut yang besar dan semakin bertumbuh, di mana banyak dari mereka bergantung pada sumber daya maritimnya. IOD telah terbukti memiliki dampak signifikan pada keanekaragaman hayati laut, perikanan, dan masyarakat pesisir, terutama selama peristiwa ekstrem seperti kekeringan \cite{Pan2018} dan siklon \cite{Wahiduzzaman2022}. Frekuensi dan intensitas peristiwa ini semakin meningkat dari hari ke hari karena pengaruh perubahan iklim. Oleh karena itu, penting untuk memahami mekanisme yang mendorong variasi IOD agar dapat mengembangkan strategi adaptasi dan mitigasi yang efektif.
	
	Selain itu, Samudera Hindia adalah pemain utama dalam sistem iklim global, dengan keterkaitan yang kuat dengan Samudera Pasifik dan Samudera Atlantik. Perubahan di Samudera Hindia dapat memiliki dampak signifikan pada pola iklim global, terutama melalui pengaruhnya pada sistem musim hujan yang menyediakan air dan makanan bagi miliaran orang di Asia dan Afrika. Oleh karena itu, memahami interaksi kompleks antara IOD dan parameter-parameter oseanografi dan meteorologi lainnya sangat penting untuk proyeksi dan pengembangan model iklim.
	
	Wilayah Samudera Hindia saat ini mengalami perubahan lingkungan yang cepat akibat aktivitas manusia dan perubahan iklim. Perubahan ini mempengaruhi sifat fisik dan kimia laut, seperti temperatur laut, salinitas, dan ketersediaan nutrisi, yang pada gilirannya mempengaruhi siklus biogeokimia dan ekosistem. IOD adalah salah satu penggerak utama dari perubahan ini, dan memahami korelasinya dengan parameter lain sangat penting untuk memprediksi dan mengurangi dampaknya pada wilayah Samudera Hindia.

	Di wilayah Indonesia sendiri, tercatat bahwa peristiwa IOD pada tahun 2019 merupakan peristiwa IOD terkuat dan terparah semenjak tahun 1980. Hal ini berdasarkan data observasi satelit dan \textit{in situ} (yaitu, kapal dan pelampung) OISSTv2 \cite{Reynolds2002}. Peristiwa ini telah menyebabkan kerugian sosial ekonomi yang besar, termasuk kekeringan parah yang berlangsung lama di Indonesia dan curah hujan tinggi di Afrika timur selama musim hujan yang singkat \cite{Bo2020}.
	
	Secara keseluruhan, penelitian tentang korelasi antara IOD dan berbagai parameter oseanografi dan meteorologi sangat mendesak untuk dilakukan karena dampak signifikan dari parameter-parameter ini pada wilayah Samudera Hindia, pola iklim global, dan kesejahteraan jutaan orang. Karena perubahan iklim terus mengubah sifat fisik dan kimia laut, memahami korelasi ini kritis untuk mengembangkan strategi adaptasi dan mitigasi yang efektif.
	\section[Manfaat Penelitian]{Manfaat Penelitian}
	
	Manfaat dan kebaruan penelitian tentang IOD terletak pada potensi untuk lebih memahami interaksi kompleks antara berbagai parameter oseanografi dan meteorologi serta dampaknya terhadap wilayah Samudera Hindia. Meskipun beberapa penelitian telah meneliti korelasi antara IOD dengan berbagai parameter oseanografi dan meteorologi, masih banyak yang harus dipelajari tentang mekanisme yang mendorong hubungan ini.
	Selain itu, karena perubahan iklim terus mengubah sifat fisik dan kimia Samudera Hindia, semakin penting untuk memahami bagaimana perubahan ini akan memengaruhi IOD dan sistem ekologi, sosial, dan ekonomi yang terkait. 
	
	Secara khusus, manfaat dari penelitian ini adalah
	\begin{itemize}
		\item Memperoleh peta IOD, arus laut, temperatur laut, salinitas, MLD, Chl-a, fluks air tawar, fluks panas bersih, laju presipitasi, dan tekanan angin di Samudera Hindia.
		\item Peta dan hasil analisis model musiman untuk parameter-parameter oseanografi dan meteorologi, seperti arus laut, temperatur laut, salinitas, MLD, Chl-a, fluks air tawar, fluks panas bersih, laju presipitasi, dan tekanan angin di Samudera Hindia.
		\item Hasil analisis korelasi untuk mengukur kekuatan dan arah dari hubungan antara IOD dengan parameter-parameter oseanografi dan meteorologi.
	\end{itemize}
	
	Adapun target jenis luaran dari penelitian ini dapat dilihat dalam Tabel \ref{tab:luaran}
	% Please add the following required packages to your document preamble:
	% \usepackage{multirow}
	\begin{table}[htp]
		\centering
		\caption{Jenis luaran dan indikator capaian tiap tahun}
		\label{tab:luaran}
		\begin{tabular}{|l|l|lll|}
			\hline
			\multirow{2}{*}{No} & \multirow{2}{*}{Jenis Luaran}                                                                                       & \multicolumn{3}{l|}{Indikator Capaian}                                                                                                                                        \\ \cline{3-5} 
			&                                                                                                                     & \multicolumn{1}{l|}{TS1} & \multicolumn{1}{l|}{TS+1}                                                         & TS+2                                                           \\ \hline
			1                   & \begin{tabular}[c]{@{}l@{}}Artikel ilmiah dimuat di \\ jurnal indeks bereputasi (JIB) \\ internasional\end{tabular} & \multicolumn{1}{l|}{}    & \multicolumn{1}{l|}{\begin{tabular}[c]{@{}l@{}}2 draft/\\ submitted\end{tabular}} & \begin{tabular}[c]{@{}l@{}}2 reviewed/\\ accepted\end{tabular} \\ \hline
			2                   & \begin{tabular}[c]{@{}l@{}}Artikel ilmiah dimuat di \\ prosiding internasional\end{tabular}                         & \multicolumn{1}{l|}{}    & \multicolumn{1}{l|}{\begin{tabular}[c]{@{}l@{}}1 draft/\\ submitted\end{tabular}} & \begin{tabular}[c]{@{}l@{}}1 reviewed/\\ accepted\end{tabular} \\ \hline
			3                   & Disertasi                                                                                                           & \multicolumn{1}{l|}{}    & \multicolumn{1}{l|}{}                                                             & 1                                                              \\ \hline
		\end{tabular}
	\end{table}
%	\section[Sistematika Disertasi]{Sistematika Disertasi}
%
%	Penelitian disertasi ini menggunakan gaya penulisan \textit{working chapter}. Setiap publikasi yang dihasilkan disajikan dalam bentuk bab. Terdapat em
%	Tesis ini tersusun atas 5 bab. Bab pertama menjelaskan pendahuluan tentang latar belakang mengapa penelitian ini dilakukan, background masalah yang mendasari, tujuan penelitian, manfaat penelitian, serta kebaruan dari penelitian. Bab kedua berisikan tinjauan pustaka menyangkut ulasan singkat materi penelitian. Bab ketiga membahas tentang metode penelitian yang dilakukan, data yang yang digunakan, serta diagram alir (\textit{flowchart}) dari penelitian. Bab keempat membahas hasil dan pembahasan penelitian. Terakhir, bab kelima membahas tentang kesimpulan dari penelitian.
	
\end{spacing}