%\vspace{1.5pc}
\vspace{1.5pc}
%\section[State of the Art]{State of the Art}
\vspace{-1pc}
\section[Penelitian Terkait IOD]{Penelitian Terkait IOD}
\begin{spacing}{1.5}
	Fenomena mode dipol (IOD) di Samudera Hindia pertama kali diteliti oleh \citeA{Saji1999}. \citeA{Saji1999} menemukan bahwa terdapat pola variasi internal dengan anomali \textit{sea surface temperature} (SST) rendah di sekitar Sumatra dan tinggi di sebelah barat Samudera Hindia, disertai dengan angin dan presipitasi. Keterkaitan spasial-temporal antara SST dan angin mengungkapkan hubungan yang kuat melalui medan presipitasi dan dinamika lautan. Proses interaksi udara-laut ini unik dan terbukti independen dari fenomena osilasi selatan El Nino (ENSO). Penemuan mode dipol ini menjelaskan sekitar 12\% variasi SST di Samudera Hindia, yang juga menyebabkan curah hujan yang parah di Afrika timur dan kekeringan di Indonesia selama tahun-tahun aktifnya.
	
	\citeA{Liu2023} menunjukkan bahwa terdapat anomali salinitas positif yang signifikan di lapisan atas (\textit{upper layer}) Samudera Hindia tropis tengah (\textit{central tropical Indian Ocean}) dari musim gugur 2010 hingga musim semi 2011 dan dari musim gugur 2016 hingga musim semi 2017. Pengaruh fenomena La Nina dan IOD negatif (nIOD) yang kuat pada tahun 2010 dan 2016 memunculkan anomali salinitas positif di Samudera Hindia timur pada akhir tahun 2010 dan 2016 sebagai hasil dari angin barat yang kuat dan arus zonal positif. \citeA{Chu2022} mengkaji tentang dinamika variabilitas antar-tahunan dari arus khatulistiwa di Samudra Hindia dan mengukur pengaruh dari fenomena ENSO dan IOD terhadap arus tersebut. ENSO dan IOD memiliki pengaruh yang sebanding terhadap anomali arus antar-tahunan di \textit{subsurface-eastern basin} (SEB), namun IOD memiliki peran yang lebih besar daripada ENSO di \textit{upper-central basin} (UCB). 
	
	\citeA{Xing2022} mengeksplorasi respons arus samudera khatulistiwa selama fase puncak IOD. Selama musim puncak IOD, serangkaian respons laut bermunculan, yaitu: (1) divergensi meridional yang signifikan di lapisan permukaan dan konvergensi di lapisan bawah permukaan di wilayah khatulistiwa dan (2) divergensi dan konvergensi meridional terbukti mendukung pelemahan \textit{Wyrtki jet} (WJ) di lapisan permukaan dan peningkatan \textit{Equatorial Undercurrent} (EUC) di lapisan bawah permukaan. \citeA{Zhang2021} menemukan bahwa peningkatan curah hujan di Samudera Hindia tropis bagian barat selama IOD positif (pIOD) melemahkan pertukaran angin timur di Samudera Atlantik tropis dan menyebabkan anomali suhu hangat di cekungan Atlantik khatulistiwa bagian tengah dan timur (\textit{central and eastern equatorial Atlantic basin}) sehingga memicu fenomena Atlantik Nino.
	
	\citeA{Polonsky2021} mengkaji fitur pembentukan lapisan kritis di zona ekuator-tropika Samudera Hindia dan kaitannya dengan IOD. Sebagian besar kasus pembentukan IOD sebagai mode inheren (tidak tergantung pada proses di Pasifik) disertai dengan pembentukan lapisan kritis. Lebih lanjut, ketidakstabilan dalam sistem arus zonal dapat menjadi alasan terjadinya fenomena IOD dan asimetri amplitudo indeks mode dipol (DMI) antara peristiwa positif dan negatif. \citeA{Valsala2020} menggunakan data pengamatan biogeo kimia dan model sirkulasi biogeo kimia laut global untuk meneliti dampak IOD pada siklus karbon di laut dan variasinya di Samudera Hindia. IOD menyebabkan variasi signifikan dalam proses pertukaran CO$_2$ dari laut ke udara di wilayah tenggara tropika Samudera Hindia (\textit{southeastern tropical Indian Ocean}) karena dinamika \textit{upwelling} dan anomali yang bergerak ke barat. \citeA{Zhang2020} menemukan fenomena \textit{Indian Ocean tripole} (IOT), yang memiliki karakteristik \textit{sea surface temperature anomalies} (SSTA) positif di wilayah tengah tropika dan negatif di tenggara dan barat Samudera Hindia. Perbedaan spasial-temporal, serta perbedaan signifikan dari hasil analisis \textit{empirical orthogonal function} (EOF) mengilustrasikan bahwa IOT berbeda dari IOD. 
	
	\citeA{Sun2022} meneliti respons asimetris SSS yang signifikan terhadap dua kejadian pIOD dan nIOD di selatan Samudera Hindia tropis. Beberapa studi lain juga menyoroti pentingnya IOD dalam menggerakkan variasi antartahunan SSS, seperti penelitian oleh \citeA{Rathore2020} yang menggunakan komposit musiman selama peristiwa ENSO/IOD untuk memahami variasi dalam transportasi kelembaban dan curah hujan di Australia, serta asosiasi mereka dengan variasi SSS. Studi lainnya oleh \citeA{Sun2019} dan \citeA{Zhang2016} mengidentifikasi mode dipol salinitas di Samudera Hindia tropis, yang disebut S-IOD, pola variasi SSS antar tahunan dengan anomali salinitas rendah di bagian tengah khatulistiwa dan salinitas tinggi di sebelah tenggara Samudera Hindia tropis.
	
	Selain itu, terdapat penelitian tentang dampak IOD pada kedalaman lapisan campuran (MLD) di Samudera Hindia, seperti penelitian oleh \citeA{Sadhukhan2021} yang menemukan bahwa peristiwa nIOD berasosiasi dengan MLD yang lebih dalam di BoB sedangkan pIOD menyebabkan MLD yang lebih dangkal. Korelasi parsial menunjukkan bahwa fluks panas bersih (NHF) adalah kontributor utama pendalaman MLD di BoB utara, sedangkan tekanan angin mengontrol pendalaman di BoB selatan. \citeA{Zhang2022} menunjukkan bahwa selama nIOD, MLD menurun karena daerah anomali evaporasi minus presipitasi negatif. Sebaliknya, selama pIOD, MLD meningkat karena daerah anomali evaporasi minus presipitasi positif. 
	
	\citeA{Sun2019} menemukan bahwa sirkulasi laut di Samudera Hindia selatan tropis berkontribusi secara signifikan terhadap anomali SSS selama peristiwa nIOD. Kenaikan gelombang Rossby membuat kedalaman termoklin dan MLD semakin dangkal. Air dengan salinitas tinggi di bawah permukaan bergerak ke lapisan permukaan dan mendinginkan SST. Hal ini menekan presipitasi lokal untuk memberikan umpan balik positif bagi peningkatan SSS. \citeA{Dandapat2021} menemukan variasi antartahunan dari MLD berhubungan dengan peristiwa IOD di BoB. MLD dangkal selama fase pIOD, sedangkan MLD rata-rata lebih dalam (sekitar 50 m) selama fase nIOD. Pada tahun pIOD, gelombang Kelvin pesisir yang tidak normal (KWs) menyebar, memantulkan gelombang Rossby, dan memicu \textit{upwelling} untuk membentuk MLD dangkal di seluruh BoB. Fluks panas bersih positif pada antarmuka udara-laut juga memainkan peran dominan dalam pendangkalan MLD pada tahun pIOD, karena radiasi gelombang pendek meningkat dan melebihi efek pendinginan fluks panas laten (LHF) selama periode ini. 
	
	\citeA{Sari2020} menemukan bahwa selama peristiwa pIOD kanonik, konsentrasi Chl-a yang tinggi diamati di sekitar Selat Sunda dan sepanjang pantai ujung barat Pulau Jawa di sekitar wilayah Cilacap. Namun, selama peristiwa pIOD Modoki, konsentrasi Chl-a lebih tinggi dan terdistribusi lebih luas. \citeA{Devi2017} menemukan bahwa pIOD menyebabkan konsentrasi Chl-a yang rendah (<2 mg/m$^3$) dan produktivitas primer yang rendah di Laut Arab (AS). El Nino menyebabkan proses \textit{downwelling}, yang mengakibatkan konsentrasi Chl-a rendah (<1 mg/m$^3$ ) di BoB dan AS. La Nina menyebabkan proses \textit{upwelling}, dan menghasilkan konsentrasi Chl-a yang tinggi (>2,0 mg/m$^3$ ) di BoB dan AS. \citeA{Mandal2022} dan \citeA{Simanjuntak2022} membahas pengaruh ENSO dan IOD pada variasi Chl-a di pesisir selatan Jawa dan pantai selatan Pulau Sunda Kecil (LSI) dan menemukan bahwa Chl-a yang intens diamati selama tahun-tahun pIOD, sedangkan konsentrasi Chl-a paling sedikit diamati selama tahun-tahun nIOD.
	
	\citeA{Luang-on2022} menunjukkan bahwa konsentrasi Chl-a di Teluk Thailand bagian atas (uGoT) terkait dengan fenomena ENSO, bukan IOD. Selama \textit{southwest monsoon} (SWM), anomali Chl-a berkorelasi dengan curah hujan dan debit sungai untuk peristiwa La Nina/El Nino. Selama musim \textit{northeast monsoon} (NEM), anomali Chl-a berkorelasi dengan debit sungai dan angin untuk peristiwa La Nina/El Nino. Sedangkan pada musim lainnya, anomali Chl-a berkorelasi dengan kecepatan angin dan curah hujan tinggi untuk peristiwa El Nino. \citeA{Setiawan2020} meneliti hubungan antara konsentrasi Chl-a, SST, dan tekanan angin permukaan laut di Laut Halmahera (HS) yang dipengaruhi oleh Monsun Australia-Indonesia (AIM), ENSO, dan IOD. Pada skala waktu antar tahunan, tekanan permukaan laut dan tekanan angin koheren dengan fase ENSO dan IOD, dan selama peristiwa El Nino dan pIOD, tekanan permukaan laut dan tekanan angin di HS sangat meningkat, sebaliknya selama peristiwa La Nina dan peristiwa nIOD, tekanan permukaan laut dan tekanan angin  di HS menurun. Hal ini mendukung peningkatan dan pengurangan konsentrasi Chl-a di wilayah tersebut. Penelitian ini menunjukkan bahwa tekanan permukaan laut dan tekanan angin sangat penting dalam menentukan konsentrasi Chl-a di HS.
	
	Secara keseluruhan, dari penelitian-penelitian yang telah disebutkan diatas, kebaruan penelitian ini terletak pada potensi untuk lebih memahami interaksi yang kompleks antara berbagai parameter dan dampaknya di kawasan Samudera Hindia. Analisis pengaruh skala spasial dan temporal yang berbeda, termasuk perbedaan regional dan tren jangka panjang juga merupakan kebaruan dalam penelitian ini. Sementara beberapa penelitian telah meneliti korelasi antara IOD dan berbagai parameter oseanografi serta paramater meteorologi masih banyak yang harus dipelajari tentang mekanisme yang mendorong hubungan ini.
	
\end{spacing}
\vspace{-1pc}
\section[Model Numerik dan Parameter]{Model Numerik dan Parameter}
\begin{spacing}{1.5}
	\par Dalam subbab ini, akan dibahas mengenai persamaan untuk parameter dan deskripsi tentang model numerik yang digunakan dalam penelitian ini.
	\subsection[Arus Laut, Temperatur Laut, dan Salinitas]{Arus Laut, Temperatur Laut, dan Salinitas}
	
	Model sirkulasi laut atau \textit{Ocean General Circulation Models} (OGCM) menggunakan persamaan Navier-Stokes untuk memodelkan fenomena fisis yang terjadi di lautan. Lautan adalah fluida yang dapat dijelaskan dengan baik dengan pendekatan persamaan-persamaan primitif, yaitu persamaan Navier-Stokes serta persamaan keadaan nonlinier yang menggabungkan dua parameter (temperatur dan salinitas) dengan kecepatan fluida, dan mempertimbangkan beberapa asumsi dan hipotesis \shortcite{madec_gurvan_2022_6334656}.
	
	Beberapa asumsi yang digunakan dalam persamaan Navier-Stokes diantaranya asumsi Boussinesq, asumsi hidrostatik, dan asumsi tak termampatkan (\textit{incompressibility}). Misalkan $\rho$ sebagai densitas in situ, $T$ sebagai temperatur potensial, $S$ sebagai salinitas, $p$ sebagai tekanan, $z$ sebagai koordinat vertikal, dan $g$ sebagai percepatan gravitasi. Asumsi yang digunakan dalam persamaan Navier-Stokes dapat dituliskan sebagai berikut.\\
	Asumsi Boussinesq
	\begin{equation}\label{eq:P1}
		\rho = \rho(T,S,p).
	\end{equation}
	Berdasarkan asumsi Boussinesq, pengaruh variasi densitas terhadap sistem diabaikan kecuali kontribusinya terhadap gaya apung.\\
	Asumsi hidrostatik
	\begin{equation}
		\frac{\partial p}{\partial z} = -\rho g.
	\end{equation}
	Berdasarkan asumsi hidrostatik, persamaan momentum vertikal direduksi menjadi persamaan kesetimbangan antara parameter gradien tekanan vertikal dan gaya apung.\\
	Asumsi tak termampatkan
	\begin{equation}
		\nabla \;.\; U =\frac{\partial u}{\partial x} + \frac{\partial v}{\partial y} + \frac{\partial w}{\partial z} = 0.
	\end{equation}	
	Berdasarkan asumsi tak termampatkan, persamaan 3-D divergensi untuk vektor kecepatan $U = (u,v,w)$ (dalam koordinat kartesius $(x,y,z)$) dianggap sama dengan 0.
	
	Selanjutnya misalkan $U = U_h + wk$ ($h$ adalah notasi vektor horizontal lokal di atas bidang $(i,j)$). Persamaan vektor invarian (invarian di bawah transformasi koordinat sehingga dapat diterapkan secara seragam dalam sistem koordinat lengkung ortogonal mana pun) dari persamaan primitif dalam sistem vektor $(i, j, k)$ dapat dituliskan dalam persamaan berikut \shortcite{madec_gurvan_2022_6334656}.\\
	Persamaan kesetimbangan momentum
	\begin{equation}\label{eq:P2}
		\begin{aligned}
			\frac{\partial U_h}{\partial t} = - \left[(\nabla \times U) \times U + \frac{1}{2}\nabla (U^2)\right]_h - f \; k \times U_h - \frac{1}{\rho_o}\nabla_h p + D^U + F^U.
		\end{aligned}
	\end{equation}
	Dalam Persamaan (\ref{eq:P2}) di atas, suku $(\nabla \times U) \times U + \frac{1}{2}\nabla (U^2)$ dapat ditulis sebagai $U\cdot \nabla U$ dan merupakan suku percepatan konvektif dari persamaan momentum. Suku $\nabla_h p$ merupakan gradien tekanan, $f = 2\Omega\; \cdot \;k$ merupakan percepatan Coriolis (dengan $\Omega$ adalah vector kecepatan sudut bumi), $D^U$ merupakan parameterisasi dari fisika skala kecil untuk momentum sedangkan $F^U$ merupakan suku gaya permukaan untuk momentum.\\
	Persamaan konservasi panas dan salinitas
	\begin{equation}\label{eq:P3}
		\begin{aligned}
			\frac{\partial T}{\partial t} &= - \nabla \; . \; (T\;U)  + D^T + F^T \\
			\frac{\partial S}{\partial t} &= - \nabla \; . \; (S\;U)  + D^S + F^S,
		\end{aligned}
	\end{equation}
	dengan operator $\nabla$ sebagai vektor turunan yang diperumum dalam arah $(i,j,k)$, parameter $D^T$ dan $D^S$ merupakan parameterisasi dari fisika skala kecil untuk temperatur dan salinitas sedangkan parameter $F^T$ dan $F^S$ merupakan suku gaya permukaan untuk temperatur dan salinitas. 
	
	\subsection[Klorofil-a dan Kedalaman Lapisan Campuran]{Klorofil-a dan Kedalaman Lapisan Campuran}
	Klorofil-a (Chl-a) adalah pigmen fotosintetik utama yang ditemukan pada tanaman, alga, dan bakteri fotosintetik. Chl-a di laut merupakan pigmen yang paling banyak ditemukan pada fitoplankton (mikroorganisme fotosintetik yang membentuk dasar dari rantai makanan laut) dan memainkan peran penting dalam siklus karbon dan oksigen di laut. Selain itu, Chl-a juga digunakan sebagai indikator kualitas air laut karena konsentrasinya dapat memberikan petunjuk tentang produktivitas di perairan tersebut. Data Chl-a yang digunakan dalam penelitian ini merupakan produk hasil dari model biogeo kimia PISCES \cite{gmd-8-2465-2015} dan merupakan bagian dari model OGCM, NEMO. 
	
	Persamaan untuk biomassa Chl-a ($I^{chl}$) (dimana $I$ dapat berupa \textit{Pytoplankton} (P) atau \textit{Diatom} (D)) untuk kedua kelompok fitoplankton diparameterisasi menggunakan model foto-adaptif \citeA{geider1997dynamic}:
	\begin{equation}\label{eq:chl1}
		\begin{aligned}
			\frac{\partial I^{chl}}{\partial t} &= (1-\delta ^I)(12\theta^{chl}_{min}+(\theta^{chl,I}_{max}-\theta^{chl,I}_{min})\rho^{I^{chl}})\mu^I I-m^I \frac{I}{K_m +I}I^chl \\
			&- sh \times w^I II^{chl}-\theta^{chl,I}g^Z(I)Z-\theta^{chl,I}g^M(I)M,
		\end{aligned}
	\end{equation}
	dengan $I$ adalah kelompok fitoplankton dan $\theta^{chl,I}$ adalah rasio klorofil-ke-karbon dari kelas fitoplankton, 12 mewakili massa molar karbon, $\rho^{I^{chl}}$ adalah rasio energi terasimilasi terhadap energi yang diserap.
	
	Kedalaman lapisan campuran atau MLD dapat dihitung dengan menggunakan temperatur laut, salinitas, atau densitas laut. MLD yang dihitung dengan temperatur dapat ditemukan pada kedalaman laut dengan temperatur yang relatif konstan. Lapisan campuran terbentuk karena adanya pengaruh angin permukaan, gelombang, dan arus yang menyebabkan pencampuran air pada lapisan atas dan membagikan panas ke seluruh lapisan ini. Di bawah lapisan campuran terdapat perubahan temperatur yang cepat seiring dengan peningkatan kedalaman laut, lapisan ini dikenal sebagai termoklin. MLD dapat diestimasi dari kombinasi variabel ambang batas dari profil densitas dan profil temperatur laut 0.2$^o$C (Pers. \ref{eq:chl2}) \cite{Boyer2004}. 
	Kriteria variabel dalam densitas sesuai dengan variasi temperatur lokal sebesar 0.2$^o$C dari temperatur pada kedalaman 10 meter (Pers. \ref{eq:chl3}). Hasil MLD akhir adalah nilai minimum dari MLD yang dihitung dari kriteria kepadatan dan MLD dari kriteria temperatur. 
	\begin{equation}\label{eq:chl2}
		\begin{aligned}
			MLD= \text{Kedalaman dimana }(\sigma_0=\sigma_{0_{10m}}+\Delta \sigma_0)
		\end{aligned}
	\end{equation}
	\begin{equation}\label{eq:chl3}
		\begin{aligned}
			MLD= \text{Kedalaman dimana }(T=T_{10m}+\Delta \pm 0.2^oC),
		\end{aligned}
	\end{equation}
	dengan $\Delta \sigma_0=\sigma_0(\theta_{10m}-0.2^oC,S_{10m},P_0)-\sigma_0(\theta_{10m},S_{10m},P_0)$.
	\subsection[Fluks Panas Bersih dan Fluks Air Tawar]{Fluks Panas Bersih dan Fluks Air Tawar}
	
	Fluks panas bersih atau \textit{net heat flux} (NHF) adalah jumlah panas yang masuk atau keluar dari permukaan laut yang disebabkan oleh perbedaan temperatur air laut dan temperatur udara di atasnya. Jika temperatur udara lebih tinggi dari temperatur air laut, maka NHF akan masuk ke laut. Sebaliknya, jika temperatur udara lebih rendah dari temperatur air laut, maka NHF akan keluar dari laut. NHF dapat dihitung sebagai jumlah dari komponen-komponen berikut \cite{Tomita2021}: 
	\begin{equation}\label{eq:NHF}
		NHF = SWR + LWR + LHF + SHF,
	\end{equation} 
	dengan SWR=\textit{net shortwave radiation}, LWR=\textit{net long wave radiation}, LHF=\textit{surface latent heat flux}, dan SHF=\textit{sensible heat flux}. Dalam penelitian ini, semua aliran panas diasumsikan positif ketika mereka mengarah ke atas, menjauhi permukaan laut ke atmosfer.
	
	Fluks air tawar atau \textit{freshwater flux} (FWF) adalah jumlah air tawar yang masuk atau keluar dari wilayah laut tertentu, termasuk curah hujan, aliran sungai, penguapan, dan es laut. Ketidakseimbangan antara air tawar masuk dan keluar dari wilayah laut dapat berdampak pada sirkulasi laut dan ketersediaan nutrisi bagi organisme laut. FWF dapat dihitung dengan cara sebagai berikut \cite{Tomita2019}:
	\begin{equation}\label{eq:FWF}\\
		\begin{aligned}
		FWF &= EVAP-RAIN, \\
		EVAP &= LHF/{\rho L_e}.
		\end{aligned}
	\end{equation} 
	dengan $EVAP$=evaporasi dan $RAIN$=curah hujan atau presipitasi di atas laut. Sedangkan $\rho$ adalah densitas air laut dan $L_e$ adalah  vaporisasi panas laten dalam air yang didefinisikan sebagai fungsi dari SST.
	
	\subsection[Ketinggian Permukaan Laut dan Arus Pasang Surut]{Ketinggian Permukaan Laut dan Arus Pasang Surut}
	Pasang surut adalah fenomena teratur dan dapat diprediksi yang disebabkan oleh tarikan gravitasi bulan dan matahari. Untuk model pasang surut, variabel yang diminati adalah arus pasang surut U (komponen timur-barat) dan V (komponen utara-selatan) serta ketinggian permukaan laut (Z). Gerakan pasang surut air laut diberikan oleh persamaan matematika berikut \cite{wahyudi2019numerical}.
	\begin{equation}\label{eq:uvtide}\\
		\begin{aligned}
			Z(t) = Z_0 \sum_{k=1}^{m}f_kH_k \cos(\omega_k t+v_k-g_k),
		\end{aligned}
	\end{equation} 
	dengan $Z(t)$ adalah tinggi permukaan air pada waktu $t$, $Z_0$ adalah tinggi permukaan air rata-rata, $f_k$ adalah faktor koreksi astronomi untuk amplitudo elemen pembangkit pasang surut $H_k$, $\omega_k$ adalah kecepatan sudut, $v_k$ adalah faktor koreksi untuk fase, dan $g_k$ adalah fase.
	
	
	\subsection[Parameter Meteorologi]{Parameter Meteorologi}
	Laju presipitasi merupakan tingkat curah hujan yang jatuh ke permukaan laut pada suatu wilayah tertentu. Hal ini dipengaruhi oleh faktor-faktor seperti SST, kelembaban udara, dan angin. Laju presipitasi di laut dapat berpengaruh pada ketersediaan air tawar, salinitas, serta pola sirkulasi dan transportasi massa air di dalam laut. Perhitungan laju presipitasi mengikuti Pers. \ref{eq:FWF}. 
	
	Angin memainkan peran penting dalam mendorong arus permukaan dan juga dalam proses interaksi udara-laut. Sebagian besar proses interaksi udara-laut ditentukan dengan menggunakan tekanan angin (\textit{wind stress}), suatu ukuran transfer momentum yang disebabkan oleh gerakan relatif antara laut dan atmosfer \shortcite{Chacko2022}. Tekanan angin berdasarkan aerodinamis massal (\textit{bulk-aerodynamic}) dirumuskan sebagai,
	\begin{equation}
		\tau = \rho_a C_d U_w^2
	\end{equation}
	dengan $\rho_a$ adalah densitas udara $(1.225 kg/m^3)$, $C_d$ koefisien seret (\textit{dimensionless}) $(\approx 1.3 \times 10^-3)$ dan $U_w$ kecepatan angin.
	
	Parameter meteorologi lainnya adalah 2m \textit{air temperature} atau temperatur udara 2m adalah temperatur udara yang berada pada 2 meter di atas permukaan (laut atau tanah). 2m \textit{specific humidity} atau kelembapan spesifik 2m mengacu pada berat (jumlah) uap air yang terkandung dalam satuan berat (jumlah) udara yang berada pada 2 meter di atas permukaan (laut atau tanah). \textit{Convective precipitation rate} atau laju presipitasi konvektif adalah laju presipitasi yang dihasilkan oleh skema konveksi. Presipitasi konvektif terjadi ketika udara naik secara vertikal melalui mekanisme konveksi mandiri (berlangsung secara singkat). Terakhir, \textit{sea level pressure} atau tekanan permukaan laut adalah tekanan atmosfer pada permukaan laut.
	
%	Dalam aplikasinya, persamaan Navier-Stokes tidak hanya digunakan untuk memodelkan laut, tapi juga merambah ke bidang pemodelan cuaca \shortcite{Rohli2021}, aliran air dalam pipa \shortcite{Ouchiha2012} dan aliran udara di sekitar sayap pesawat \shortcite{Tulus2019}. Dalam bentuk persamaan lengkap dan simplifikasi, persamaan ini juga dapat digunakan untuk mendesain kereta api \shortcite{Croquer2020}, pesawat terbang \shortcite{Chau2021}, dan mobil \shortcite{Ambarita2018}. Terdapat juga studi tentang aliran darah \shortcite{Gill2021}, desain stasiun pembangkit listrik \shortcite{Yang2019}, dan analisis polusi udara \shortcite{Issakhov2022}. 
	
\end{spacing}
\vspace{-1pc}
\section[Peta Jalan Penelitian]{Peta Jalan Penelitian}
\begin{spacing}{1.5}
	\textit{Road Map} atau peta jalan penelitian ini dapat dilihat dalam Gambar \ref{fig:RM}
	\begin{figure}[H]
		\centering
		\includegraphics[width=10cm]{contents/Figures/Road_Map}
		\caption{\textit{Road Map} Penelitian}
		\label{fig:RM}
	\end{figure}
	
\end{spacing}