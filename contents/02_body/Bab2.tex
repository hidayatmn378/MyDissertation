%\vspace{1.5pc}
\vspace{1.5pc}
%\section[State of the Art]{State of the Art}
\vspace{-1pc}
\section[Penelitian Terkait dan Tahapan Penelitian IOD]{Penelitian Terkait dan Tahapan Penelitian IOD}
\begin{spacing}{1.5}
	Berdasarkan hasil analisis data observasi selama 40 tahun (1958-1997), \citeA{Saji1999} menunjukkan suatu fenomena mode dipol di Samudera Hindia, yaitu pola variasi internal dengan suhu permukaan laut (\textit{sea surface temperature} (SST)) anomali yang rendah di sekitar Sumatra dan SST yang tinggi di sebelah barat Samudera Hindia, dengan angin dan presipitasi yang menyertainya. Keterkaitan spasial-temporal antara SST dan angin mengungkapkan keterkaitan yang kuat melalui medan presipitasi dan dinamika laut. Proses interaksi udara-laut ini unik dan melekat pada Samudera Hindia, dan terbukti independen dari fenomena Osilasi Selatan El Nino (El Nino \textit{Southern Oscillation} (ENSO)). Penemuan mode dipol ini yang menjelaskan sekitar 12\% dari variasi SST di Samudera Hindia, dan pada tahun-tahun aktifnya juga menyebabkan curah hujan yang parah di Afrika Timur dan kekeringan di Indonesia. 
	
	\citeA{Liu2023} menemukan anomali salinitas positif yang signifikan di lapisan atas Samudra Hindia tropis pusat selama periode tertentu dan menganalisis kondisi atmosfer dan proses dinamika samudra yang terkait dengan peristiwa ini. \citeA{Chu2022} mempelajari dinamika variasi antartahunan arus khatulistiwa di Samudra Hindia dan mengukur efek dari mode iklim ENSO dan IOD pada arus. \citeA{Xing2022} mengeksplorasi respons arus laut khatulistiwa selama fase puncak IOD, yang memberikan umpan balik positif laut yang mendukung puncak IOD. \citeA{Zhang2021} meneliti Atlantic Nino dan dampaknya pada iklim regional dan global. 
	
	\citeA{Polonsky2021} mengkaji fitur pembentukan lapisan kritis di zona ekuator-tropika Samudra Hindia, dan bagaimana hal tersebut terkait dengan terbentuknya IOD. \citeA{Valsala2020} meneliti dampak IOD pada siklus karbon di atas laut dan variasinya di Samudra Hindia, dengan pengamatan biogeo kimia dan model sirkulasi biogeo kimia laut global. Mereka menemukan bahwa IOD menyebabkan variasi signifikan dalam pengeluaran CO2 dari laut ke udara di wilayah tenggara tropika Samudra Hindia karena dinamika upwelling dan anomali bergerak ke barat. \citeA{Zhang2020} membedakan IOD dari pola tripole baru yang baru saja ditemukan, yang memiliki \textit{sea surface temperature anomalies} (SSTA) positif (negatif) di atas wilayah tengah tropika (tenggara dan barat) Samudra Hindia. Studi-studi ini berkontribusi pada pemahaman kita tentang interaksi kompleks antara arus laut, kondisi atmosfer, dan variabilitas iklim di Samudra Hindia dan Atlantik.
	
	Dalam kaitannya dengan SST dan \textit{sea surface salinity} (SSS). \citeA{Akhil2023} menemukan bahwa penyegaran permukaan laut di bagian tenggara Laut Arab (\textit{southeastern Arabian Sea} (SEAS)) selama musim dingin dipicu oleh adveksi horisontal air tawar \textit{Bay of Bengal} (BoB) oleh sirkulasi siklonik di sekitar India selama musim gugur, dan IOD adalah penggerak utama dari variasi antartahunan SSS di SEAS selama musim dingin. Namun, dampak penyegaran SEAS musim dingin pada SST lokal dan awal musim hujan berikutnya lemah. \citeA{Genda2022} menemukan bahwa sebelum pertengahan 1950-an, SST bervariasi dengan IOD, sedangkan ENSO juga mempengaruhi variasi SST setelah pertengahan 1950-an. Variasi SSS tidak menunjukkan hubungan dengan faktor-faktor iklim, mengindikasikan bahwa faktor-faktor pengontrol utama SST dan SSS harus dipertimbangkan secara terpisah. \citeA{Sun2022} meneliti respons asimetris SSS terhadap dua kejadian IOD positif dan negatif ekstrim di Selatan Tropis Samudra Hindia dan menemukan asimetri yang signifikan dalam SSS selama kejadian IOD positif dan negatif.
	
	\citeA{Rathore2020} menggunakan komposit musiman selama peristiwa ENSO/IOD untuk memahami variasi dalam transportasi kelembaban dan curah hujan di atas Australia, dan asosiasi mereka dengan variasi SSS. \citeA{Sun2019} menyelidiki variasi SSS di sebelah barat daya Samudra Hindia tropis (SWTIO) yang terkait dengan peristiwa IOD negatif tahun 2010 dan mengidentifikasi proses kunci yang berkontribusi pada variasi SSS di SWTIO. \citeA{Zhang2016} mengidentifikasi mode dipol keasinan di Samudra Hindia tropis, yang disebut S-IOD, pola variasi SSS antar tahunan dengan keasinan anomali rendah di sebelah tengah khatulistiwa dan keasinan tinggi di sebelah tenggara Samudra Hindia tropis. Secara keseluruhan, studi-studi ini menyoroti pentingnya IOD dalam menggerakkan variasi antartahunan SSS, tetapi juga menunjukkan bahwa SST dipengaruhi oleh faktor-faktor iklim lainnya dan faktor pengontrol utama SST dan SSS harus dipertimbangkan secara terpisah.
	
	\citeA{Sadhukhan2021} melakukan penelitian terhadap dampak IOD pada MLD di Laut Arab selatan timur menggunakan simulasi model laut dan menemukan bahwa peristiwa IOD positif menyebabkan MLD yang lebih dalam. \citeA{Zhang2022} berfokus pada dampak IOD pada MLD di Samudera Hindia khatulistiwa bagian timur, dan penelitian mereka menunjukkan bahwa peristiwa IOD negatif menyebabkan peningkatan MLD karena efek pendinginan yang lebih besar. \citeA{Sayantani2014} mengusulkan mekanisme alternatif untuk inisiasi IOD yang melibatkan pemanasan musim semi Laut Arab dan lapisan penghalang musim panas yang terkait. Terakhir, \citeA{Sun2019} menyelidiki variasi salinitas permukaan laut (SSS) dan hubungannya dengan dinamika laut di Samudera Hindia tropis selatan barat yang terkait dengan peristiwa IOD negatif tahun 2010. Mereka menemukan bahwa sirkulasi laut di Samudera Hindia selatan tropis berkontribusi secara signifikan terhadap anomali SSS selama evolusi peristiwa IOD negatif, dan gelombang Rossby upwelling membuat kedalaman termoklin dan MLD dangkal, membawa air subpermukaan berkepadatan tinggi ke lapisan permukaan dan mendinginkan suhu permukaan laut, yang lebih menekan presipitasi lokal untuk memberikan umpan balik positif bagi peningkatan SSS. Studi-studi tersebut menunjukkan bahwa peristiwa IOD memiliki dampak yang signifikan pada kedalaman lapisan campuran di berbagai wilayah Samudera Hindia, dan pemahaman tentang hubungan ini sangat penting untuk memprediksi dan mengelola efek dari variasi iklim di wilayah tersebut.
	
	\citeA{Sari2020} menemukan bahwa selama peristiwa IOD positif kanonik, konsentrasi chl-a yang tinggi diamati di sekitar Selat Sunda dan sepanjang pantai ujung barat Pulau Jawa di sekitar wilayah Cilacap. Sementara itu, selama peristiwa IOD positif Modoki, konsentrasi chl-a yang relatif lebih tinggi dan lebih terdistribusi luas daripada yang diamati selama peristiwa IOD positif kanonik. Analisis menunjukkan bahwa peristiwa upwelling yang relatif lemah yang ditunjukkan oleh kedalaman lapisan isotermal yang dalam (ILD) selama peristiwa IOD positif Modoki yang dikombinasikan dengan ketebalan lapisan penghalang yang tipis (BLT) dan lapisan campuran yang dalam memberikan kondisi yang menguntungkan untuk peningkatan konsentrasi chl-a di wilayah Southeastern Tropical Indian Ocean (SETIO). Sementara itu, upwelling yang kuat yang ditunjukkan oleh kedalaman ILD yang dangkal yang dikombinasikan dengan BLT yang tebal dan lapisan campuran yang dangkal mencegah peningkatan konsentrasi chl-a selama peristiwa IOD positif kanonik. Di sisi lain, \citeA{Devi2017} menemukan bahwa IOD positif menyebabkan konsentrasi chl-a yang rendah (<2 mg/m$^3$) dan produktivitas primer yang rendah di Laut Arab (AS). \citeA{Mandal2022} dan \citeA{Simanjuntak2022} membahas pengaruh ENSO dan IOD pada variasi chl-a, sementara \citeA{Luang-on2022} memfokuskan pada variasi musiman dan antar tahun konsentrasi chl-a di Teluk Thailand bagian atas. \citeA{Setiawan2020} menyelidiki hubungan antara konsentrasi chl-a, SST, dan tekanan angin permukaan laut di Laut Halmahera yang dipengaruhi oleh Monsun Australia-Indonesia (AIM), ENSO, dan IOD.
	
	
\end{spacing}
\vspace{-1pc}
\section[Model Numerik dan Parameter]{Model Numerik dan Parameter}
\begin{spacing}{1.5}
	\par Dalam subbab ini, akan dibahas mengenai persamaan untuk parameter dan deskripsi tentang model numerik yang digunakan dalam penelitian ini.
	
	Model sirkulasi laut atau \textit{Ocean General Circulation Models} (OGCM) menggunakan persamaan Navier-Stokes untuk memodelkan fenomena fisis yang terjadi di lautan. Lautan adalah fluida yang dapat dijelaskan dengan baik dengan pendekatan persamaan-persamaan primitif, yaitu persamaan Navier-Stokes serta persamaan keadaan nonlinier yang menggabungkan dua parameter (temperatur dan salinitas) dengan kecepatan fluida, dan mempertimbangkan beberapa asumsi dan hipotesis \shortcite{madec_gurvan_2022_6334656}.
	
	Beberapa asumsi yang digunakan dalam persamaan Navier-Stokes diantaranya asumsi Boussinesq, asumsi hidrostatik, dan asumsi tak termampatkan (\textit{incompressibility}). Misalkan $\rho$ sebagai densitas in situ, $T$ sebagai temperatur potensial, $S$ sebagai salinitas, $p$ sebagai tekanan, $z$ sebagai koordinat vertikal, dan $g$ sebagai percepatan gravitasi. Asumsi yang digunakan dalam persamaan Navier-Stokes dapat dituliskan sebagai berikut.\\
	Asumsi Boussinesq
	\begin{equation}\label{eq:P1}
		\rho = \rho(T,S,p).
	\end{equation}
	Berdasarkan asumsi Boussinesq, pengaruh variasi densitas terhadap sistem diabaikan kecuali kontribusinya terhadap gaya apung.\\
	Asumsi hidrostatik
	\begin{equation}
		\frac{\partial p}{\partial z} = -\rho g.
	\end{equation}
	Berdasarkan asumsi hidrostatik, persamaan momentum vertikal direduksi menjadi persamaan kesetimbangan antara parameter gradien tekanan vertikal dan gaya apung.\\
	Asumsi tak termampatkan
	\begin{equation}
		\nabla \;.\; U =\frac{\partial u}{\partial x} + \frac{\partial v}{\partial y} + \frac{\partial w}{\partial z} = 0.
	\end{equation}	
	Berdasarkan asumsi tak termampatkan, persamaan 3-D divergensi untuk vektor kecepatan $U = (u,v,w)$ (dalam koordinat kartesius $(x,y,z)$) dianggap sama dengan 0.
	
	Selanjutnya misalkan $U = U_h + wk$ ($h$ adalah notasi vektor horizontal lokal di atas bidang $(i,j)$). Persamaan vektor invarian (invarian di bawah transformasi koordinat sehingga dapat diterapkan secara seragam dalam sistem koordinat lengkung ortogonal mana pun) dari persamaan primitif dalam sistem vektor $(i, j, k)$ dapat dituliskan dalam persamaan berikut \shortcite{madec_gurvan_2022_6334656}.\\
	Persamaan kesetimbangan momentum
	\begin{equation}\label{eq:P2}
		\begin{aligned}
			\frac{\partial U_h}{\partial t} = - \left[(\nabla \times U) \times U + \frac{1}{2}\nabla (U^2)\right]_h - f \; k \times U_h - \frac{1}{\rho_o}\nabla_h p + D^U + F^U.
		\end{aligned}
	\end{equation}
	Dalam Persamaan (\ref{eq:P2}) di atas, suku $(\nabla \times U) \times U + \frac{1}{2}\nabla (U^2)$ dapat ditulis sebagai $U\cdot \nabla U$ dan merupakan suku percepatan konvektif dari persamaan momentum. Suku $\nabla_h p$ merupakan gradien tekanan, $f = 2\Omega\; \cdot \;k$ merupakan percepatan Coriolis (dengan $\Omega$ adalah vector kecepatan sudut bumi), $D^U$ merupakan parameterisasi dari fisika skala kecil untuk momentum sedangkan $F^U$ merupakan suku gaya permukaan untuk momentum.\\
	Persamaan konservasi panas dan salinitas
	\begin{equation}\label{eq:P3}
		\begin{aligned}
			\frac{\partial T}{\partial t} &= - \nabla \; . \; (T\;U)  + D^T + F^T \\
			\frac{\partial S}{\partial t} &= - \nabla \; . \; (S\;U)  + D^S + F^S,
		\end{aligned}
	\end{equation}
	dengan operator $\nabla$ sebagai vektor turunan yang diperumum dalam arah $(i,j,k)$, parameter $D^T$ dan $D^S$ merupakan parameterisasi dari fisika skala kecil untuk temperatur dan salinitas sedangkan parameter $F^T$ dan $F^S$ merupakan suku gaya permukaan untuk temperatur dan salinitas. 
	
	Untuk parameter lainnya seperti kedalaman lapisan campuran atau \textit{mixed layer depth} (MLD), \textit{chlorophyll-a} (Chl-a), laju presipitasi, fluks air tawar, fluks panas bersih, dan tekanan angin dapat dituliskan sebagai berikut.
	
%	Dalam aplikasinya, persamaan Navier-Stokes tidak hanya digunakan untuk memodelkan laut, tapi juga merambah ke bidang pemodelan cuaca \shortcite{Rohli2021}, aliran air dalam pipa \shortcite{Ouchiha2012} dan aliran udara di sekitar sayap pesawat \shortcite{Tulus2019}. Dalam bentuk persamaan lengkap dan simplifikasi, persamaan ini juga dapat digunakan untuk mendesain kereta api \shortcite{Croquer2020}, pesawat terbang \shortcite{Chau2021}, dan mobil \shortcite{Ambarita2018}. Terdapat juga studi tentang aliran darah \shortcite{Gill2021}, desain stasiun pembangkit listrik \shortcite{Yang2019}, dan analisis polusi udara \shortcite{Issakhov2022}. 
	
\end{spacing}
\vspace{-1pc}
\section[\textit{Road Map }Penelitian]{\textit{Road Map} Penelitian}
\begin{spacing}{1.5}
	\textit{Road Map} penelitian ini dapat dilihat dalam Gambar \ref{fig:RM}
	\begin{figure}[H]
		\centering
		\includegraphics[width=12cm]{contents/Figures/Road_Map}
		\caption{Road Map Penelitian}
		\label{fig:RM}
	\end{figure}
	
\end{spacing}