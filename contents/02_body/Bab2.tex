%\vspace{1.5pc}
\vspace{1.5pc}
%\section[State of the Art]{State of the Art}
\begin{spacing}{1.5}
	
	Bab ini menjelaskan lebih detail mengenai pustaka relevan dan tinjauan teori dalam penelitian ini. Hal ini bertujuan untuk mereview, mengupdate, mengkritik dan mensintesis literatur, melakukan meta-analisis literatur, melakukan konsepsi ulang dari topik yang direview, dan menjawab pertanyaan spesifik penelitian dari topik yang telah direview dalam literatur \shortcite{Torraco2016}.
	
\end{spacing}
\vspace{-1pc}
\section[Penelitian Terkait dan Tahapan Penelitian IOD]{Penelitian Terkait dan Tahapan Penelitian IOD}
\begin{spacing}{1.5}
	\textit{Indian Ocean Dipole} (IOD) adalah6
\end{spacing}
\vspace{-1pc}
\section[Persamaan Model Numerik]{Persamaan Model Numerik}
\begin{spacing}{1.5}
	\par Model sirkulasi laut atau \textit{Ocean General Circulation Models} (OGCM) menggunakan persamaan Navier-Stokes untuk memodelkan fenomena fisis yang terjadi di lautan. Lautan adalah fluida yang dapat dijelaskan dengan baik dengan pendekatan persamaan-persamaan primitif, yaitu persamaan Navier-Stokes serta persamaan keadaan nonlinier yang menggabungkan dua variabel (temperatur dan salinitas) dengan kecepatan fluida, dan mempertimbangkan beberapa asumsi dan hipotesis \shortcite{madec_gurvan_2022_6334656}.
	
	Beberapa asumsi yang digunakan dalam persamaan Navier-Stokes diantaranya asumsi Boussinesq, asumsi hidrostatik, dan asumsi tak termampatkan (\textit{incompressibility}). Misalkan $\rho$ sebagai densitas in situ, $T$ sebagai temperatur potensial, $S$ sebagai salinitas, $p$ sebagai tekanan, $z$ sebagai koordinat vertikal, dan $g$ sebagai percepatan gravitasi. Asumsi yang digunakan dalam persamaan Navier-Stokes dapat dituliskan sebagai berikut.\\
	Asumsi Boussinesq
	\begin{equation}\label{eq:P1}
		\rho = \rho(T,S,p).
	\end{equation}
	Berdasarkan asumsi Boussinesq, pengaruh variasi densitas terhadap sistem diabaikan kecuali kontribusinya terhadap gaya apung.\\
	Asumsi hidrostatik
	\begin{equation}
		\frac{\partial p}{\partial z} = -\rho g.
	\end{equation}
	Berdasarkan asumsi hidrostatik, persamaan momentum vertikal direduksi menjadi persamaan kesetimbangan antara variabel gradien tekanan vertikal dan gaya apung.\\
	Asumsi tak termampatkan
	\begin{equation}
		\nabla \;.\; U =\frac{\partial u}{\partial x} + \frac{\partial v}{\partial y} + \frac{\partial w}{\partial z} = 0.
	\end{equation}	
	Berdasarkan asumsi tak termampatkan, persamaan 3-D divergensi untuk vektor kecepatan $U = (u,v,w)$ (dalam koordinat kartesius $(x,y,z)$) dianggap sama dengan 0.
	
	Selanjutnya misalkan $U = U_h + wk$ ($h$ adalah notasi vektor horizontal lokal di atas bidang $(i,j)$). Persamaan vektor invarian (invarian di bawah transformasi koordinat sehingga dapat diterapkan secara seragam dalam sistem koordinat lengkung ortogonal mana pun) dari persamaan primitif dalam sistem vektor $(i, j, k)$ dapat dituliskan dalam persamaan berikut \shortcite{madec_gurvan_2022_6334656}.\\
	Persamaan kesetimbangan momentum
	\begin{equation}\label{eq:P2}
		\begin{aligned}
			\frac{\partial U_h}{\partial t} = - \left[(\nabla \times U) \times U + \frac{1}{2}\nabla (U^2)\right]_h - f \; k \times U_h - \frac{1}{\rho_o}\nabla_h p + D^U + F^U.
		\end{aligned}
	\end{equation}
	Dalam Persamaan (\ref{eq:P2}) di atas, suku $(\nabla \times U) \times U + \frac{1}{2}\nabla (U^2)$ dapat ditulis sebagai $U\cdot \nabla U$ dan merupakan suku percepatan konvektif dari persamaan momentum. Suku $\nabla_h p$ merupakan gradien tekanan, $f = 2\Omega\; \cdot \;k$ merupakan percepatan Coriolis (dengan $\Omega$ adalah vector kecepatan sudut bumi), $D^U$ merupakan parameterisasi dari fisika skala kecil untuk momentum sedangkan $F^U$ merupakan suku gaya permukaan untuk momentum.\\
	Persamaan konservasi panas dan salinitas
	\begin{equation}\label{eq:P3}
		\begin{aligned}
			\frac{\partial T}{\partial t} &= - \nabla \; . \; (T\;U)  + D^T + F^T \\
			\frac{\partial S}{\partial t} &= - \nabla \; . \; (S\;U)  + D^S + F^S,
		\end{aligned}
	\end{equation}
	dengan operator $\nabla$ sebagai vektor turunan yang diperumum dalam arah $(i,j,k)$, variabel $D^T$ dan $D^S$ merupakan parameterisasi dari fisika skala kecil untuk temperatur dan salinitas sedangkan variabel $F^T$ dan $F^S$ merupakan suku gaya permukaan untuk temperatur dan salinitas. 
	 
	Dalam aplikasinya, persamaan Navier-Stokes tidak hanya digunakan untuk memodelkan laut, tapi juga merambah ke bidang pemodelan cuaca \shortcite{Rohli2021}, aliran air dalam pipa \shortcite{Ouchiha2012} dan aliran udara di sekitar sayap pesawat \shortcite{Tulus2019}. Dalam bentuk persamaan lengkap dan simplifikasi, persamaan ini juga dapat digunakan untuk mendesain kereta api \shortcite{Croquer2020}, pesawat terbang \shortcite{Chau2021}, dan mobil \shortcite{Ambarita2018}. Terdapat juga studi tentang aliran darah \shortcite{Gill2021}, desain stasiun pembangkit listrik \shortcite{Yang2019}, dan analisis polusi udara \shortcite{Issakhov2022}. 
	\subsection[CMEMS]{CMEMS}
	\subsection[HYCOM]{HYCOM}
	\subsection[J-OFURO3]{J-OFURO3}
	\subsection[NCEP/NCAR]{NCEP/NCAR}
	
\end{spacing}
\vspace{-0.1pc}