\vspace{1.5pc}
\section[Domain dan Data Penelitian]{Domain dan Data Penelitian}
\begin{spacing}{1.5}
	\subsection[Domain Penelitian]{Domain Penelitian}
	Penelitian ini dilakukan di Samudera Hindia dengan koordinat ($0^\circ-24.6^\circ$ N) dan ($78.2^\circ-105^\circ$ E) (lihat Gambar \ref{fig:domain}).

	\subsection[Data Penelitian]{Data Penelitian}
%	\vspace{-1pc}
%		\subsection[NEMO/CMEMS]{NEMO/CMEMS}
%	\par Model NEMO (\textit{Nucleus for European Modelling of the Ocean}) (\href{https://www.nemo-ocean.eu/}{https://www.nemo-ocean.eu/}) adalah model komputasi resolusi tinggi yang terus dikembangkan sejak tahun 2008 oleh konsorsium Eropa yang terdiri dari 5 institusi, yaitu CMCC, CNRS, \textit{Mercator Ocean}, \textit{Met Office}, dan NERC. Model ini digunakan untuk penelitian dan peramalan dalam bidang oseanografi dan klimatologi, dengan tujuan menjadi alat yang fleksibel untuk mempelajari fenomena fisik dan biogeokimia dalam sirkulasi laut, serta interaksinya dengan komponen sistem iklim Bumi, pada skala ruang dan waktu yang berbeda  \shortcite{madec_gurvan_2022_6334656}. Data \textit{output} model NEMO dapat diperoleh dari website CMEMS (\textit{Copernicus Marine Environment Monitoring Service}) \href{https://resources.marine.copernicus.eu/products}{(https://resources.marine.copernicus.eu/products)}. CMEMS merupakan w
%	
%	Data yang tersedia pada website CMEMS dapat dilihat pada Tabel . Resolusi data \textit{output} yang digunakan untuk model ini adalah dx = dy = 5 menit pada bidang horizontal dan 50-lapisan $(k \in [1,50])$ dengan ketebalan berbeda pada bidang vertikal:
%	\begin{equation*}
%		\begin{aligned}
%			z_k = \{0.49, 1.54, 2.65, 3.82, 5.08, 6.44, 7.93, 9.57, 11.40, 13.47, 15.82, 18.50, \\
%			21.60, 25.21, 29.44, 34.43, 40.34, 47.37, 55.76, 65.81, 77.85, 92.33, 109.73, 130.67, \\
%			155.85, 186.12, 222.47, 266.04, 318.13, 380.21, 453.94, 541.089, 643.57, 763.33, \\
%			902.34, 1062.44, 1245.29, 1452.25, 1684.28, 1941.89, 2225.08, 2533.33, 2865.70,  \\
%			3220.82, 3597.03, 3992.48, 4405.22, 4833.29, 5274.78, 5727.92 } (m). \\
%		\end{aligned}
%	\end{equation*}
%	\subsection[HYCOM]{HYCOM}
%	\par Model HYCOM (\textit{HYbrid Coordinate Ocean Model}) (\href{https://www.hycom.org}{https://www.hycom.org}) adalah salah satu model sirkulasi laut (OGCM) yang menggunakan model numerik tiga dimensi Navier-Stokes dengan input data batimetri dari GEBCO (\textit{General Bathymetric Chart of the Oceans}), data asimilasi hidrografi laut dari NCODA (\textit{Navy Coupled Ocean Data Assimilation}) dan komponen meteorologi dari NCEP (\textit{National Centers for Environmental Prediction}) ataupun NAVGEM (\textit{The NAVy Global Environmental Model}) berupa tekanan angin, kecepatan, fluks panas, tekanan permukaan laut, presipitasi, temperatur laut, dan kelembapan \shortcite{JosephMetzger2013}. 
%	

%	Koordinat vertikal dalam HYCOM adalah isopiknal di lautan terbuka yang terstratifikasi dan memiliki transisi yang mulus dan dinamis serta bergantung terhadap waktu pada medan daerah pesisir yang dangkal dan pada tingkat tekanan tetap di lapisan campuran permukaan atau lautan yang tidak terstratifikasi \shortcite{chassignet2017,Park2013}. Data HYCOM yang digunakan adalah data analisis global arus dan temperatur laut tiga dimensi dengan resolusi spasial 5 menit untuk longitude dan 2.5 menit untuk latitude selama 12 bulan (Januari - Desember) tahun 2021 dan dengan ketebalan bervariasi pada bidang vertikal, yaitu 40-lapisan $(k \in [1,40])$:
%	\begin{equation*}
%		\begin{aligned}
%			z_k = \{0.0, 2.0, 4.0, 6.0, 8.0, 10.0, 12.0, 15.0, 20.0, 25.0, 30.0, 35.0, 40.0, 45.0, 50.0, \\
%			60.0, 70.0,	80.0, 90.0, 100.0, 125.0, 150.0, 200.0, 250.0, 300.0, 350.0, 400.0, 500.0, 600.0,\\
%			700.0, 800.0, 900.0, 1000.0, 1250.0, 1500.0, 2000.0, 2500.0, 3000.0, 4000.0, 5000.0} (m). \\
%		\end{aligned}
%	\end{equation*}
%	\subsection[J-OFURO3]{J-OFURO3}
%	\subsection[NCEP/NCAR]{NCEP/NCAR}
	Tabel \ref*{tab:data} adalah rangkuman data yang digunakan dalam penelitian ini. Data DMI diperoleh dari \textit{National Oceanic and Atmospheric Administration/Physical Sciences Laboratory} (NOAA/PSL) \cite{Saji2003} sedangkan data arus laut, temperatur laut, salinitas, MLD, dan Chl-a dapat diperoleh dari website penyedia data \textit{Copernicus Marine Environment Monitoring Service} (CMEMS)\cite{Lellouche2018} dan \textit{HYbrid Coordinate Ocean Model} (HYCOM)\cite{Chassignet2007}. Data lainnya adalah fluks air tawar, fluks panas bersih, laju presipitasi, dan tekanan angin yang bersumber dari \textit{National Centers for Environmental Prediction and the National Center for Atmospheric Research reanalysis 1} (NCEP/NCAR) \cite{Kalnay1996} dan J-OFURO3 yang merupakan generasi ketiga dari \textit{Japanese ocean flux data set} yang menggunakan pengamatan penginderaan jarak jauh \cite{Tomita2019}. 
	
%	Hasil yang sangat baik diperoleh dengan membandingkan data temperatur laut 3D (dalam Kelvin) CMEMS dengan data pengamatan \textit{in situ} pada kedalaman 0 hingga 5 m di Samudera Hindia. Hal ini tercermin dari nilai \textit{Root Mean Square} (RMS), untuk perbandingan \textit{in situ thematic centre} (INS TAC), dengan nilai 0.65, dan untuk data \textit{in situ} CORIOLIS, dengan nilai 0.44. Perbandingan data salinitas 3D (dalam satuan salinitas praktis (psu)) dengan data pengamatan \textit{in situ} juga menunjukkan hasil yang sangat baik. Nilai RMS pada kedalaman 0 hingga 5 m untuk data global dan Samudera Hindia masing-masing adalah 0.65 dan 0.204 (Lellouche et al., 2019). Sama halnya dengan perbandingan model CMEMS dan konsentrasi \textit{float} Chl-a (\textit{Biogeochemical-Argo} (BGC-Argo), data pengukuran). Hal ini ditunjukkan oleh koefisien korelasi dan \textit{Root Mean Square Error} (RMSE) masing-masing sebesar 0,81 dan 0,59 (Lamouroux et al., 2019).   
	
	\begin{table}[H]
		\centering
		\caption{Rangkuman data penelitian}
		\label{tab:data}
		\resizebox{\textwidth}{!}{%
		\begin{tabular}{|lllll|}
			\hline
			\multicolumn{1}{|l|}{No} & \multicolumn{1}{l|}{Data}                                                                                     & \multicolumn{1}{l|}{Periode}   & \multicolumn{1}{l|}{Sumber}                                                             & Referensi                                            \\ \hline
			\multicolumn{5}{|c|}{Data IOD}                                                                                                                                                                                                                                                                                             \\ \hline
			\multicolumn{1}{|l|}{1}  & \multicolumn{1}{l|}{DMI}                                                                                      & \multicolumn{1}{l|}{1994-2021} & \multicolumn{1}{l|}{NOAA/PSL}                                                           & \cite{Saji2003}                     \\ \hline
			\multicolumn{5}{|c|}{Data Oseanografi}                                                                                                                                                                                                                                                                                     \\ \hline
			\multicolumn{1}{|l|}{2}  & \multicolumn{1}{l|}{Arus laut}                                                                                & \multicolumn{1}{l|}{1994-2021} & \multicolumn{1}{l|}{CMEMS/HYCOM}                                                        & \cite{Lellouche2018,Chassignet2007} \\ \hline
			\multicolumn{1}{|l|}{3}  & \multicolumn{1}{l|}{Temperatur laut}                                                                          & \multicolumn{1}{l|}{1994-2021} & \multicolumn{1}{l|}{CMEMS/HYCOM}                                                        & \cite{Lellouche2018,Chassignet2007} \\ \hline
			\multicolumn{1}{|l|}{4}  & \multicolumn{1}{l|}{Salinitas}                                                                                & \multicolumn{1}{l|}{1994-2021} & \multicolumn{1}{l|}{CMEMS/HYCOM}                                                        & \cite{Lellouche2018,Chassignet2007} \\ \hline
			\multicolumn{1}{|l|}{5}  & \multicolumn{1}{l|}{MLD}                                                                                      & \multicolumn{1}{l|}{1994-2021} & \multicolumn{1}{l|}{CMEMS}                                                              & \cite{Lellouche2018}                \\ \hline
			\multicolumn{1}{|l|}{6}  & \multicolumn{1}{l|}{Klorofil-a (Chl-a)}                                                                       & \multicolumn{1}{l|}{1994-2021} & \multicolumn{1}{l|}{CMEMS}                                                              & \cite{Lellouche2018}                \\ \hline
			\multicolumn{1}{|l|}{7}  & \multicolumn{1}{l|}{Fluks air tawar}                                                                          & \multicolumn{1}{l|}{1994-2017} & \multicolumn{1}{l|}{J-OFURO3}                                                           & \cite{Tomita2019}                   \\ \hline
			\multicolumn{1}{|l|}{8}  & \multicolumn{1}{l|}{Fluks panas bersih}                                                                       & \multicolumn{1}{l|}{1994-2017} & \multicolumn{1}{l|}{J-OFURO3}                                                           & \cite{Tomita2019}                   \\ \hline
			\multicolumn{1}{|l|}{9}  & \multicolumn{1}{l|}{\begin{tabular}[c]{@{}l@{}}Ketinggian \\ permukaan laut (SL)\end{tabular}}                & \multicolumn{1}{l|}{2022}      & \multicolumn{1}{l|}{TMD}                                                                & \cite{padman2005tide}               \\ \hline
			\multicolumn{1}{|l|}{10} & \multicolumn{1}{l|}{Arus pasang surut}                                                                        & \multicolumn{1}{l|}{2022}      & \multicolumn{1}{l|}{TMD}                                                                & \cite{padman2005tide}               \\ \hline
			\multicolumn{5}{|c|}{Data Meteorologi}                                                                                                                                                                                                                                                                                     \\ \hline
			\multicolumn{1}{|l|}{11} & \multicolumn{1}{l|}{Laju presipitasi}                                                                         & \multicolumn{1}{l|}{1994-2021} & \multicolumn{1}{l|}{NCEP/NCAR}                                                          & \cite{Kalnay1996}                   \\ \hline
			\multicolumn{1}{|l|}{12} & \multicolumn{1}{l|}{tekanan angin}                                                                            & \multicolumn{1}{l|}{1994-2021} & \multicolumn{1}{l|}{NCEP/NCAR}                                                          & \cite{Kalnay1996}                   \\ \hline
			\multicolumn{1}{|l|}{13} & \multicolumn{1}{l|}{\begin{tabular}[c]{@{}l@{}}Laju konvektif \\ presipitasi\end{tabular}}                    & \multicolumn{1}{l|}{2002-2021} & \multicolumn{1}{l|}{NCEP/NCAR}                                                          & \cite{Kalnay1996}                   \\ \hline
			\multicolumn{1}{|l|}{14} & \multicolumn{1}{l|}{Temperatur udara}                                                                         & \multicolumn{1}{l|}{2002-2021} & \multicolumn{1}{l|}{NCEP/NCAR}                                                          & \cite{Kalnay1996}                   \\ \hline
			\multicolumn{1}{|l|}{15} & \multicolumn{1}{l|}{Kelembaban spesifik}                                                                      & \multicolumn{1}{l|}{2002-2021} & \multicolumn{1}{l|}{NCEP/NCAR}                                                          & \cite{Kalnay1996}                   \\ \hline
			\multicolumn{1}{|l|}{16} & \multicolumn{1}{l|}{\begin{tabular}[c]{@{}l@{}}Tekanan permukaan \\ laut\end{tabular}}                        & \multicolumn{1}{l|}{2002-2021} & \multicolumn{1}{l|}{NCEP/NCAR}                                                          & \cite{Kalnay1996}                   \\ \hline
			\multicolumn{5}{|c|}{Data Perikanan}                                                                                                                                                                                                                                                                                       \\ \hline
			\multicolumn{1}{|l|}{17} & \multicolumn{1}{l|}{\begin{tabular}[c]{@{}l@{}}Produktifitas primer \\ bersih (NPP)\end{tabular}}             & \multicolumn{1}{l|}{2002-2021} & \multicolumn{1}{l|}{\begin{tabular}[c]{@{}l@{}}Oregon State \\ University\end{tabular}} & \cite{lan2013effects}               \\ \hline
			\multicolumn{1}{|l|}{18} & \multicolumn{1}{l|}{\begin{tabular}[c]{@{}l@{}}Data tangkapan perikanan \\ rawai jangka panjang\end{tabular}} & \multicolumn{1}{l|}{2002-2021} & \multicolumn{1}{l|}{IOTC}                                                               & \cite{lan2013effects}               \\ \hline
		\end{tabular}%
	}
	\end{table}
	\begin{figure}[H]
		\centering
		\includegraphics[width=12cm]{contents/Figures/Batimetri_edit_compress}
		\caption{Peta batimetri Samudera Hindia, diperoleh dari SRTM30+ \protect\cite{becker2009global}. Warna dalam peta menunjukkan kedalaman 0-5000 m sedangkan pulau digambarkan tanpa warna.}
		\label{fig:domain}
	\end{figure}
	\subsection[Pengumpulan Data]{Pengumpulan Data}
	Data dalam Tabel \ref{tab:data} merupakan data yang tersedia secara gratis dan bersifat terbuka. Data ini dapat diunduh secara langsung pada website penyedia data ataupun menggunakan kode skrip. Kode skrip yang digunakan untuk mengunduh data dengan bahasa \textit{Shell script} (terminal Linux) dan Python disajikan dalam Lampiran 1.
	
\end{spacing}
\vspace{-0.5pc}
\section[Analisis Data]{Analisis Data}
\begin{spacing}{1.5}
	\subsection[Model Musiman]{Model Musiman}

	Data jangka panjang dapat dianalisis dengan menggunakan analisis deret waktu, sebagai contoh analisis model musiman, yang bertujuan untuk menganalisis kemungkinan adanya pola musiman yang berulang dalam periode tertentu pada data tersebut. Monsun merupakan penyebab utama perubahan IOD, parameter oseanografi seperti arus laut, temperatur laut, salinitas, MLD, Chl-a, fluks air tawar, dan fluks panas bersih, serta parameter meteorologi seperti laju presipitasi dan angin. Penelitian ini akan menganalisis parameter-parameter tersebut dengan menggunakan model musiman yang didasarkan pada pola kejadian monsun. Model musiman ini digunakan untuk mengamati pola periodik dan memprediksi parameter-parameter berdasarkan keteraturan mereka. Dengan cara ini, akan diketahui apakah parameter yang diteliti mengikuti pola monsun yang terjadi, apakah terjadi pergeseran bulanan pada masing-masing parameter dalam model musiman yang dibangun, serta apakah parameter-parameter tersebut mengikuti pola monsun yang sama.

	Beberapa penelitian yang menggunakan model musiman adalah \shortciteauthor{Haridhi2016} \citeyear{Haridhi2016} yang meneliti tentang hubungan antara SST dan \textit{net deployment} (ND) - penyebaran jaring nelayan pukat cincin tradisional. \shortciteauthor{Ikhwan2022} \citeyear{Ikhwan2022} dalam penelitiannya mengkaji tentang MLD di Laut Andaman (AS) menggunakan data SSS dari model 3-D \textit{Copernicus Marine Environment Monitoring Service} (CMEMS).  
	
	Setiap fungsi periodik dapat diuraikan menjadi penjumlahan gelombang sinus dan kosinus. Persamaan terkait hal ini disebut juga sebagai deret Fourier atau model Harmonik. Secara matematis hal ini dapat dituliskan sebagai \cite{goela2016time,hyndman2018forecasting} 
	\begin{equation}\label{eq:fs}
		y = \alpha + \sum_{k=1}^{N} \left[ \beta_k \sin(\frac{2\pi kt}{m})+\gamma_k \cos(\frac{2\pi kt}{m})\right]  + \epsilon,
	\end{equation}
	dengan $\alpha, \beta_k, \gamma_k,k,\text{ dan }m$ adalah konstanta pergeseran vertikal, koefisien komponen sinus, koefisien komponen kosinus, ordo atau frekuensi gelombang sinus dan kosinus, dan periode fungsi. Bentuk sederhana dari Pers. \ref{eq:fs} dapat dituliskan sebagai
	persamaan siklus musiman \cite{crawley2012r}
	\begin{equation}\label{eq:sm_}
		y = \alpha + \beta \sin(2\pi t)+\gamma \cos(2\pi t) + \epsilon,
	\end{equation}
	dengan $\alpha, \beta$, dan $\gamma$  adalah konstanta pergesaran vertikal, amplitudo dari gelombang sinus, dan amplitudo dari gelombang kosinus. Dalam persamaan ini, $t$ adalah waktu dan $\epsilon$ adalah elemen residual yang mewakili komponen \textit{white-noise} tidak beraturan dalam proses pengambilan data. Parameter-parameter yang tidak diketahui, yaitu $\alpha, \beta, $ dan $\gamma$, diperkirakan menggunakan metode kuadrat terkecil (\textit{ordinary least squares}) \cite{goela2016time}.
	
	Gambar \ref{fig:sm} merupakan ilustrasi persamaan \ref{eq:sm_} untuk nilai $\alpha,\beta$ dan $\gamma$ yang berbeda. Nilai $\alpha$ yang berbeda mempengaruhi posisi kurva terhadap sumbu-y. Sedangkan nilai $\beta$ dan $\gamma$ yang berbeda mempengaruhi posisi kurva terhadap sumbu-x.
	\begin{figure}[H]
		\centering
		\includegraphics[width=15cm]{contents/Figures/sm_experiment}
		\caption{Ilustrasi persamaan model musiman.}
		\label{fig:sm}
	\end{figure}
	
%	Langkah-langkah yang digunakan untuk menggambarkan \textit{seasonal model} dapat dijelaskan sebagai berikut.
%	\begin{itemize}
%		\item \textbf{\textit{Start}}. \textit{Software} yang digunakan untuk menentukan nilai $\alpha, \beta$ dan $\gamma$ adalah \textit{software} \textit{R}.
%		\item \textbf{\textit{Initialization}}. Dalam proses inisialisasi ini, komponen musiman semua parameter dimodelkan untuk mendapatkan siklus tahunan.
%		\item \textbf{\textit{Input}}. Setelah proses inisialisasi \textit{input} parameter yang ingin diprediksi kemudian dimasukkan.
%		\item \textbf{\textit{Process}}. Selanjutnya dilakukan perhitungan dengan menggunakan fungsi model linear (\verb|lm()|) dalam \textit{R}.
%		\item \textbf{\textit{Output}}. Perintah Summary() digunakan untuk menampilkan hasil dalam proses sebelumnya. Dalam tahapan ini juga diperoleh penggambaran \textit{seasonal model}.
%	\end{itemize}

	\subsection[Analisis Korelasi]{Analisis Korelasi}
		Analisis korelasi berfungsi sebagai alat eksplorasi data dan pembuatan hipotesis dalam menentukan hubungan antara parameter yang diteliti. Dalam penelitian ini, analisis korelasi digunakan untuk mengukur kekuatan (nilai koefisien korelasi) dan arah (positif/negatif) hubungan antara indeks IOD dengan parameter lainnya, serta untuk menentukan signifikansi koefisien korelasi tersebut.
		
		Analisis korelasi mempelajari hubungan linier antara variabel/parameter yang diteliti. Jika terdapat hubungan antara dua karakteristik, langkah selanjutnya adalah mengevaluasi apakah variabel pertama dapat digunakan untuk memprediksi variabel kedua melalui analisis regresi. Kekuatan dari korelasi ditentukan berdasarkan koefisien korelasi ($R$), yang bervariasi antara -1 dan +1. Berdasarkan \citeA{schober2018correlation}, jika $R\in[0.00,0.10]$ maka korelasi diabaikan. Jika $R\in[0.10,0.39]$ maka korelasi lemah. Jika $R\in[0.40,0.69]$ maka korelasi kuat. Jika $R\in[0.90,1.00]$ maka korelasi sangat kuat. Sementara itu, arah dari korelasi dapat berupa korelasi positif atau korelasi negatif. Korelasi positif terjadi jika nilai variabel pertama berbanding lurus dengan variabel kedua. Sedangkan korelasi negatif terjadi jika nilai variabel pertama berbanding terbalik dengan variabel kedua. Persamaan koefisien korelasi dapat dituliskan sebagai berikut  \cite{hidayat2023relationship,Haditiar2020,zhao2022spearman}.
		\begin{itemize}
			\item Korelasi Pearson
			\begin{equation}
				\begin{aligned}
					R &= \frac{\sum (x_i - \bar{x})(y_i - \bar{y})}{\sqrt{\sum (x_i-\bar{x})^2\sum (y_i-\bar{y})^2}},
				\end{aligned}
			\end{equation}
			dengan $x_i, y_i$ adalah variable yang digunakan untuk menghitung koefisien korelasi dengan $i$ adalah indeks data. Sedangkan $\bar{x}$ dan $\bar{y}$ adalah rata-rata. Analisis korelasi Pearson merupakan korelasi linear antara variabel yang berukuran metrik.
			\item Korelasi Spearman
			\begin{equation}
				\begin{aligned}
					%				y &= b+ax\\
					R_s &= 1-\frac{6\cdot\sum d_i^2}{n\cdot(n^2-1)},
				\end{aligned}
			\end{equation}
			dengan $n$ adalah jumlah kasus dan $d_i$ adalah perbedaan peringkat antara dua variabel. Korelasi Spearman adalah pendamping non-parametrik dari korelasi Pearson.
		\end{itemize}
	
		Untuk menggunakan korelasi Pearson, variabel harus terdistribusi secara normal dan harus terdapat hubungan linear antara variabel-variabel tersebut. Distribusi normal dapat diuji baik secara analitis maupun grafis dengan menggunakan plot QQ (\textit{Quantile-Quantile plot}) sedangkan korelasi linear dapat diperiksa dengan menggunakan scatter plot. Jika kondisi-kondisi ini tidak terpenuhi, maka digunakan korelasi Spearman.
		
		Signifikansi dari koefisien korelasi diuji dengan uji t dengan cara menguji apakah koefisien korelasi secara signifikan berbeda dari nol (uji bebas linier). Dalam hal ini, hipotesis nol menyatakan tidak ada korelasi antara variabel yang dipertimbangkan, sementara hipotesis alternatif mengasumsikan adanya korelasi. Dengan tingkat signifikansi $(\alpha)$ pada 1\% atau 5\%, jika nilai p < 5\%, hipotesis nol ditolak dan hipotesis alternatif diterima yang berarti bahwa terdapat hubungan antara variabel dalam populasi. Persamaan untuk t-statistik dapat dituliskan sebagai
		\begin{equation}
			\begin{aligned}
				t=\frac{R\sqrt{n-2}}{1-R^2}.
			\end{aligned}
		\end{equation}
		dengan $n$ ukuran sampel dan $R$ korelasi yang ditentukan dalam sampel.
		\subsection[Analisis Regresi]{Analisis Regresi}
		Analisis regresi memungkinkan pemodelan hubungan antara variabel dependen dan satu atau lebih variabel independen. Dengan analisis regresi, dapat diketahui ukuran pengaruh satu atau lebih variabel terhadap variabel lain serta dapat memprediksi sebuah variabel berdasarkan satu atau lebih variabel lainnya. Hal ini karena analisis regresi memberikan informasi tentang bagaimana nilai variabel dependen berubah jika salah satu variabel independen diubah. Dalam hal ini, variabel dependen ($y$) dapat berupa IOD, Chl-a, atau data perikanan, sedangkan variabel independen ($x$) dapat berupa arus laut, temperatur laut, salinitas, MLD, fluks air tawar, fluks panas bersih, ketinggian permukaan laut, arus pasang surut, laju presipitasi, dan tekanan angin.  Analisis regresi dibedakan sebagai berikut  \cite{ANGELINI2019722}.
		\begin{itemize}
			\item Regresi Linear Sederhana
			\begin{equation}
				\begin{aligned}
					y &= a \cdot x+b\\
				\end{aligned}
			\end{equation}
			dengan $y$ dan $x$ adalah variabel dependen dan independen, $b$ adalah konstanta titik potong sumbu-$y$, dan $a$ adalah koefisien regresi yang mengukur pengaruh variabel independen $(x)$ terhadap variabel dependen $(y)$.
			\item Regresi Linier Berganda
			\begin{equation}
				\begin{aligned}
					y &= a_1 \cdot x_1+a_2 \cdot x_2+\dots+a_k \cdot x_k+b,
				\end{aligned}
			\end{equation}
			dengan $y$ dan $x_1,x_2,\dots,x_k$ adalah variabel dependen dan independen, $b$ adalah konstanta titik potong sumbu-$y$, dan $a_1,a_2,\dots,a_k$ adalah koefisien regresi yang mengukur pengaruh variabel independen $(x_1,x_2,\dots,x_k)$ terhadap variabel dependen $(y)$.
			\item Regresi Polinomial
			\begin{equation}
				\begin{aligned}
					y &= a_0+a_1 \cdot x+a_2 \cdot x^2+\dots+a_k \cdot x^k,
				\end{aligned}
			\end{equation}
			dengan $y$ dan $x$ adalah variabel dependen dan independen, $k$ adalah derajat polinomial, dan $a_1,a_2,\dots,a_k$ adalah koefisien regresi yang mengukur pengaruh variabel independen $(x)$ terhadap variabel dependen $(y)$.
		\end{itemize}
	
		Untuk mengevaluasi kualitas prediksi atau penjelasan model regresi terhadap variabel dependen, digunakan dua ukuran utama, yaitu koefisien determinasi $R^2$ dan kesalahan estimasi standar (\textit{standard estimation error}). Koefisien determinasi $R^2$ menggambarkan proporsi variansi yang dapat dijelaskan oleh variabel independen. Semakin tinggi nilai $R^2$, semakin baik kualitas model regresi. Perhitungan $R^2$ melibatkan perbandingan antara variansi nilai perkiraan ($S_{\hat{y}}$) dan variansi nilai yang diamati $(S_y)$,
		\begin{equation}
			\begin{aligned}
				R^2 &= \frac{S_{\hat{y}}}{S_y}.
			\end{aligned}
		\end{equation}
	
		Koefisien determinasi $R^2$  dipengaruhi oleh jumlah variabel independen yang digunakan. Semakin banyak variabel independen yang termasuk dalam model regresi, semakin besar resolusi variansi $R^2$. Untuk memperhitungkan hal ini, digunakan $R^2$ yang disesuaikan (\textit{adjusted }$R^2$).
		\begin{equation}
			\begin{aligned}
				R^2_{adj} &= 1-(1-R^2)\cdot\frac{n-1}{n-k-1}.
			\end{aligned}
		\end{equation}
		dengan $n$ adalah jumlah sampel dan $k$ adalah jumlah variabel independen dalam model regresi. Kesalahan estimasi standar adalah simpangan baku dari kesalahan estimasi. Hal ini memberikan gambaran sejauh mana prediksi berbeda dari nilai yang sebenarnya. Secara grafis, kesalahan estimasi standar adalah penyebaran nilai yang diamati di sekitar garis regresi.
		\begin{equation}
				SE = \sqrt{\frac{\sum (y_i-\hat{y_i})^2}{n-k-1}},				
		\end{equation}
		dengan $y_i$ adalah nilai yang diamati, $\hat{y_i}$ adalah nilai yang diprediksi dari model regresi, $n$ adalah jumlah sampel, dan $k$ adalah jumlah variabel independen.
\end{spacing}
\vspace{-0.5pc}
\section[Prosedur Penelitian]{Prosedur Penelitian}
\begin{spacing}{1.5}
	Prosedur penelitian mengikuti diagram alir pada Gambar \ref{fig:flowchart} dan dapat dijelaskan sebagai berikut. 
	\begin{itemize}
		\item \textbf{\textit{Start}}. \textit{Software} yang digunakan dalam penelitian ini adalah Matlab dan \textit{R}.
		\item \textbf{\textit{Input}}. Data penelitian diunduh terlebih dahulu. Adapun data yang digunakan adalah data IOD (DMI), data oseanografi, data meteorologi, dan data perikanan berdasarkan Tabel \ref{tab:data}.
		\item \textbf{\textit{Process}}. Data kemudian diolah dengan cara melakukan pemetaan dan penggambaran data. Pada tahap ini, akan dilakukan proses pemetaan data, yaitu proses memetakan atau menetapkan data dari satu format ke format lain dengan tujuan untuk memudahkan pengolahan data, sehingga dapat digunakan untuk tujuan analisis atau pelaporan. Selanjutnya pembersihan data, yaitu proses mengidentifikasi, mengubah, atau menghapus data yang tidak akurat, tidak lengkap, atau tidak relevan dalam kumpulan data. Terakhir proses penggambaran data, yaitu proses menyajikan data dalam bentuk visual, seperti grafik, diagram, atau peta dengan tujuan untuk memudahkan pemahaman dan analisis terhadap data yang kompleks dan memudahkan untuk melihat pola dan tren dari data yang disajikan.
		
		\item \textbf{\textit{Output}}. Penggambaran data pada tahap sebelumnya menghasilkan peta IOD, oseanografi, meteorologi, dan perikanan secara jangka panjang. 
		\item \textbf{\textit{Process}}. Dalam tahap ini akan dilakukan beberapa analisis data, diantaranya model musiman (\textit{seasonal model}), analisis korelasi, dan analisis regresi.
		
%		Tahapan yang dilakukan dalam analisis \textit{seasonal model} dapat dijelaskan sebagai berikut.
%		\begin{itemize}
%			\item Input: data DMI, arus laut, temperatur laut, salinitas, MLD, Chl-a, fluks air tawar, fluks panas bersih, laju presipitasi, dan tekanan angin.
%			\item Pemodelan komponen musiman: Pada tahap ini, nilai $\alpha, \beta$, dan $\gamma$ pada Pers. \ref{eq:sm_} akan ditentukan dengan menggunakan fungsi model linear (lm()) dalam R. Nilai ini kemudian digunakan untuk memodelkan komponen musiman.
%			\item \textit{Plotting Model}: Setelah komponen musiman dimodelkan, langkah selanjutnya adalah melakukan \textit{plotting model} untuk memvisualisasikan hasil analisis.
%		\end{itemize}
		
%		Tahapan yang dilakukan dalam analisis korelasi dapat dijelaskan sebagai berikut.
%		\begin{itemize}
%			\item Tentukan hipotesis nol dan hipotesis alternatif.
%			\item Tentukan tingkat signifikansi alpha (alpha 1\% atau 5\%).
%			\item Hitung koefisien korelasi antara dua variabel.
%			\item Hitung nilai p (nilai probabilitas) dari koefisien korelasi untuk menentukan signifikansi statistik.
%			\item Bandingkan nilai p dengan tingkat signifikansi $\alpha$. Jika p < $\alpha$, maka korelasi signifikan secara statistik dan hipotesis nol dapat ditolak. Sedangkan jika p > $\alpha$, maka tidak ada bukti yang cukup untuk menolak hipotesis nol dan korelasi dianggap tidak signifikan.
%			\item Plot hasil uji korelasi.
%		\end{itemize}
		\item \textbf{\textit{Output}}. Dari proses analisis data diperoleh peta seasonal model, peta korelasi, dan tabulasi hasil.
		\item \textbf{\textit{Process}}. Hasil output dari peta-peta IOD, oseanografi, meteorologi, perikanan, model musiman, korelasi, dan tabulasi kemudian diinterpretasikan dan dituliskan dalam laporan.
		\item \textbf{\textit{End}}. Penelitian selesai.
	\end{itemize}
	\begin{figure}[H]
		\centering
		\includegraphics[width=14cm]{contents/Figures/Flowchart_Diagram.png}
		\caption{Diagram alir penelitian}
		\label{fig:flowchart}
	\end{figure}
\end{spacing}
