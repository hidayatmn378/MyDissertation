%\vspace{1.5pc}
\begin{spacing}{1.5}
	\thispagestyle{empty}
	\begin{flushleft}
		\begin{tabular}{lp{0.25cm}p{9cm}}
			1. Nama &:& Muh. Nur Hidayat\\
			2. Tempat, tanggal lahir &:& Makassar, 03 Juli 1998\\
			3. Alamat &:& Jl. Blang Bintang Lama, Babah Jurong, Kec. Kuta Baro, Kabupaten Aceh Besar, Aceh \\
			4. Nama Ayah &:& Jamaluddin\\
			5. Pekerjaan Ayah &:& -\\
			6. Nama Ibu &:& Nurlaela\\
			7. Pekerjaan Ibu &:& Ibu rumah Tangga\\	
			8. Alamat Orang Tua &:& Kota Makassar, Sulawesi Selatan\\
			9. Riwayat Pendidikan &:& \\
		\end{tabular}
	\end{flushleft}
	
	\begin{table}[H]
		\centering
		\vspace{-1cm}
		\resizebox{\columnwidth}{!}{%
			\begin{tabular}{|l|l|l|l|l|}
				\hline
				Jenjang & Nama Sekolah                & Bidang Studi & Tempat            & Tahun Ijazah \\ \hline
				SD      & SD Inp. 102 Bontokadatto    & -            & Kabupaten Takalar & 2010         \\ \hline
				SMP     & SMP Negeri 1 Mangarabombang & -            & Kabupaten Takalar & 2013         \\ \hline
				SMA     & SMA Negeri 1 Takalar        & IPA          & Kabupaten Takalar & 2016         \\ \hline
				Sarjana (S1) &
				\begin{tabular}[c]{@{}l@{}}Program Studi Matematika, Fakultas MIPA, \\ Universitas Hasanuddin\end{tabular} &
				\begin{tabular}[c]{@{}l@{}}Matematika dan \\ Aplikasinya\end{tabular} &
				Kota Makassar &
				2020 \\ \hline
				Magister (S2) &
				\begin{tabular}[c]{@{}l@{}}Program Studi Magister Matematika, Fakultas \\ MIPA, Universitas Syiah Kuala\end{tabular} &
				\begin{tabular}[c]{@{}l@{}}Matematika dan \\ Aplikasinya\end{tabular} &
				Kota Banda Aceh &
				2023 \\ \hline
			\end{tabular}%
		}
	\end{table}
	\vspace{-0.5cm}
	\noindent\hspace{0.1cm} 10. Karya tulis yang pernah dihasilkan :
	\begin{table}[H]
		\centering
		\vspace{-0.3cm}
		\resizebox{\columnwidth}{!}{%
			\begin{tabular}{|l|l|l|l|}
				\hline
				No & Judul                                                                                                     & Tahun & Penerbit               \\ \hline
				1  & \begin{tabular}[c]{@{}l@{}}MODEL MATEMATIKA TRANSPORTASI CO2 \\ DALAM PROSES KARBONASI BETON (Skripsi)\end{tabular} & 2020  & Universitas Hasanuddin \\ \hline
				2 &
				\begin{tabular}[c]{@{}l@{}}\textit{A Two-Dimensional Mathematical Model of Carbon} \\ \textit{Dioxide (CO2) Transport in Concrete Carbonation Proses}\end{tabular} &
				2021 &
				\begin{tabular}[c]{@{}l@{}}Jurnal Matematika, Statistika \\ dan Komputasi (JMSK)\end{tabular} \\ \hline
				3 &
				\begin{tabular}[c]{@{}l@{}}\textit{Relationship between chlorophyll-a, sea surface temperature,} \\ \textit{and sea surface salinity}\end{tabular} &
				2023 &
				\begin{tabular}[c]{@{}l@{}}\textit{Global Journal of Environmental} \\ \textit{Science and Management} (GJESM)\end{tabular} \\ \hline
				4 &
				\begin{tabular}[c]{@{}l@{}}PENGARUH PARAMETER METEOROLOGI TERHADAP \\ KEDALAMAN LAPISAN CAMPURAN (\textit{MIXED LAYER DEPTH})\\ DAN APLIKASINYA DIPERAIRAN SAMUDERA HINDIA (Tesis)\end{tabular} &
				2023 &
				Universitas Syiah Kuala \\ \hline
			\end{tabular}%
		}
	\end{table}
\end{spacing}
\begin{table}[H]
	\begin{tabular}{lll}
		\quad \quad \quad \quad \quad \quad \quad \quad \quad \quad \quad	& \quad \quad \quad \quad \quad \quad \quad \quad \quad \quad \quad& \begin{tabular}[c]{@{}l@{}}Banda Aceh, 15 Juni 2023\\ Yang menyatakan,\\ \\ \\ \\ Muh. Nur Hidayat\\ NPM. 2108201010005\end{tabular}
	\end{tabular}
\end{table}