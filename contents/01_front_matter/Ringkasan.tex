\begin{spacing}{1.5}
	\pagestyle{empty}
	\begin{center}
		\vskip 1cm
		\justifying
		Penelitian ini membahas tentang \textit{Indian Ocean Dipole} (IOD), yang merupakan perbedaan temperatur permukaan laut antara Samudera Hindia bagian barat tropis dan Samudera Hindia bagian timur tropis. IOD memiliki peran penting dalam sistem iklim global dan dampaknya terhadap musim hujan, pertanian, dan perikanan di wilayah tersebut. Karena meningkatnya emisi gas rumah kaca antropogenik, penelitian ini bertujuan untuk memodelkan parameter oseanografi dan meteorologi yang menggerakkan IOD untuk memahami hubungan antara IOD dan perubahan iklim. Penelitian ini menggunakan aplikasi pemodelan laut untuk menganalisis interaksi antara parameter oseanografi seperti arus, temperatur laut, salinitas, MLD, Chl-a, fluks air tawar, dan fluks panas bersih, serta parameter meteorologi seperti tekanan angin dan laju presipitasi. Tujuannya adalah untuk memberikan wawasan tentang mekanisme yang mendorong IOD dan potensi responsnya terhadap perubahan iklim. Penelitian ini diharapkan dapat berkontribusi pada pemahaman kita tentang interaksi kompleks antara laut dan atmosfer di wilayah Samudera Hindia.
	\end{center}
\end{spacing}
\pagestyle{empty}