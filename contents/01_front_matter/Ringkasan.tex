\begin{spacing}{1.5}
	\pagestyle{empty}
	\begin{center}
		\vskip 1cm
		\justifying
		Penelitian ini bertujuan untuk memahami hubungan antara parameter oseanografi, meteorologi, monsun, dan kelimpahan ikan di Samudera Hindia. Hal ini penting karena Samudera Hindia berperan dalam sistem iklim global dan perubahan di dalamnya berdampak signifikan. Penelitian ini menggunakan pemodelan laut untuk menganalisis interaksi parameter-parameter tersebut dan memberikan wawasan tentang IOD dan responsnya terhadap perubahan iklim. Penelitian ini juga menganalisis hubungan antara IOD, klorofil-a, kedalaman lapisan campuran, dan kelimpahan ikan. IOD berdampak pada iklim, pertanian, dan perikanan. Perubahan IOD mempengaruhi suhu permukaan laut, pola arus, kelimpahan plankton, dan produktivitas di laut, termasuk konsentrasi klorofil-a. Klorofil-a merupakan indikator produktivitas primer di laut dan penting bagi rantai makanan, termasuk ikan. Perubahan kedalaman lapisan campuran juga mempengaruhi nutrien, oksigen, dan kondisi fisik yang penting bagi ikan. Penelitian ini menyelidiki hubungan parameter-parameter tersebut dengan monsun, mengkaji pengaruh IOD terhadap kelimpahan dan distribusi ikan, serta menggali interaksi kompleks laut dan atmosfer di Samudera Hindia. Hasil penelitian diharapkan memberikan pemahaman mendalam tentang perubahan iklim di wilayah ini dan dasar untuk mitigasi dan adaptasi.\\
		\textbf{Kata kunci}: Samudera Hindia; Indian Ocean Dipole (IOD); Perubahan iklim; Interaksi laut dan atmosfer.
	\end{center}
\end{spacing}
\pagestyle{empty}